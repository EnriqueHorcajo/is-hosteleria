%Documento que incluye la gestion de riesgos segun el modelo BOEHM

\documentclass[spanish,a4paper,12pt]{report}	% Idioma, tamaño del papel, tamaño letra, documento (book, report, article, letter)

%%% PAQUETES
\usepackage[spanish,activeacute]{babel}				
% Babel: Adapta cosas como la tipografia, la fecha, lo de Chapter al español, y activeacute para apóstrofes (') como abreviaciones de acentos: \'{a}
\usepackage[utf8]{inputenc}					% Codificacion UTF8 (para meter tildes normal: á --> \'{a} )
\usepackage{multicol}						% Escritura en varias columnas
\usepackage{graphics}						% Inclusión de imágenes
\usepackage{graphicx}						% Mas para imagenes
\usepackage{geometry}						% Distribucion de la pagina: margenes, encabezados, tamaño pagina...
\usepackage{fancyhdr}						% Paquete para añadir y modificar encabezados y pies de pagina
\usepackage{hyperref}						% Para hipervínculos, en el indice al menos, GRACIAS A DAVID
%\usepackage{lastpage}						% Ultima pagina para poner, por ejemplo, 3 de 15
%%% PAQUETES MATEMATICOS
\usepackage{amsmath}						% Conjunto de paquetes desarrollados por la Amercian Matematical Society
\usepackage{amssymb}						% Tipografía mathbb y otros símbolos tambien de la AMS
\usepackage{amsthm}						% Paquete AMS theorem, de la AMS
\usepackage{amsfonts}						% Paquete con símbolos y mas, de la AMS
%\usepackage{nicefrac}						% Fracciones bonitas, LO DEJO COMENTADO PORQUE A VECES DA PROBLEMAS AL COMPILAR


%%% DECLARACIONES (sobre la forma de la pagina, encabezado etc.)
\pagenumbering{roman}						
% Para numerar las paginas en numeros romanos hasta que empiece el texto (tambien alph, Alph, roman, Roman...)
\pagestyle{fancy}							% Utiliza el paquete fancyhdr para encabezados y pies de pagina
%\thispagestyle{empty}  						% Para poner UNA pagina sin encabezados ni numero, "plain" CON numero, "fancy" normal
%\lhead{Encabezado a la izquierda}				% Encabezado a la izquierda
\rhead{\bfseries Gestión de riesgos}			%Encabezado a la derecha
\cfoot{\thepage}							% Numero de pagina centrado en el pie
%\cfoot{\thepage\ de \pageref{LastPage}}		% Numero de pagina centrado en el pie asi: n de m
\renewcommand{\headrulewidth}{0.4pt}			% Linea debajo del encabezado
\renewcommand{\footrulewidth}{0.4pt}			% Linea encima del pie de pagina
\renewcommand*{\thesection}{\arabic{section}}	% Hace que no apareca el indice de capitulos y que comience en section, GRACIAS A RUBEN


%%%%% CUERPO %%%%%
\begin{document}

%<<<<<<< .mine


\title{\textbf{\huge{Gestión de \\ riesgos Software}}}
%=======
\title{\textbf{\huge{Gestión de \\ 
	riesgos Software}} \\ \vspace{0.3cm}
	\Large{Ingeniería del Software}}
%>>>>>>> .r52
\author{ Jesús Aguirre Pemán \\
	 Enrique Ballesteros Horcajo \\
	 Jaime Dan Porras Rhee \\
	 Ignacio Iker Prado Rujas \\
	 Alejandro Villarín Prieto }
\date{\Today}
\maketitle



\tableofcontents 							%INDICE hipervinculado


\newpage
\mbox{}
\thispagestyle{empty}						% Hoja en blanco, sin numeros ni nada
\newpage
\setcounter{section}{0}

\part{Identificación de los riesgos}
\begin{itemize}
\item \textbf {Deficiencias del personal}
	\begin{itemize}
		\item {Baja de algun miembro del equipo}
		\item {Baja del supervisor del proyecto}
	\end{itemize}
\item \textbf {Planificaciones y presupuestos poco realistas}
	\begin{itemize}
		\item {Retraso en las entregas por mala planificación}
		\item {No entregar todo lo acordado en la planificación por falta de tiempo}
		\item {Perdidas insubsanables}
		\item {Mayores gastos de lo esperado}
		\item {Cierre del proyecto por ser insostenible}
	\end{itemize}
\item \textbf {Desarrollo de las funciones y propiedades erróneas}
	\begin{itemize}
		\item {Las funciones son ineficientes}
		\item {Poca calidad de las funciones y propiedades realizadas}
		\item {Las distintas partes del proyecto no cumplen con su cometido}
		\item {Dificultad para hacer que las distintas funciones del proyecto se coordinen entre ellas}
		\item {El producto no se ajusta a lo que el cliente necesita por falta de comunicacion}
		\item {El cliente rechaza las funciones que hemos desarrollado}
		\item {El cliente no sabe que funciones debe desarrollar el producto}
		\item {El producto no funciona debidamente en la plataforma en que se quiere usar}
		\item {El lenguaje no permite realizar todas las funciones}
	\end{itemize}
\item \textbf {Desarrollo erróneo del interfaz de usuario}
	\begin{itemize}
		\item {La interfaz de usuario es demasiado dificil de construir}
		\item {Falta de recursos para el desarrollo de la interfaz}
		\item {El cliente considera que la interfaz es dificil de usar}
		\item {Al cliente no le resulta atractiva la interfaz de usuario}
		\item {El cliente decide cambiar por completo la interfaz de usuario}
	\end{itemize}
\item \textbf {Chapado}
	\begin{itemize}
		\item {Disminucion de los funciones debido al alto coste}
		\item {Menor calidad del producto para disminuir el gasto}
		\item {Abandono del proyecto}
	\end{itemize}
\item \textbf {Continua corriente de cambios en los requisitos}
	\begin{itemize}
		\item {El cliente cambia de opinión acerca de lo que debe hacer el proyecto}
		\item {El cliente no sabe qué espera que haga el producto}
		\item {Los distintos clientes aportan visiones muy distintas del producto}

	\end{itemize}
\item \textbf {Deficiencias en componentes proporcionados externamente}
	\begin{itemize}
		\item {Las librerias de java no son eficientes para nuestro proyecto}
		\item {Los programas proporcionados son muy dificiles de usar y poco efectivos}
		\item {Los recursos son proporcionados demasiado tarde}

	\end{itemize}
\item \textbf {Deficiencias en tareas desarrolladas externamente}
	\begin{itemize}
		\item {Poco tiempo para realizar correciones}
		\item {Poco tiempo para asimilar los pasos a seguir}
	\end{itemize}
\item \textbf {Deficiencias en rendimiento en tiempo real}
	\begin{itemize}
		\item {Falta de recursos para realizar el proyecto}
		\item {Nuestro producto no cumple con los requisitos de rendimiento}
		\item {Nuestro producto no garantiza la calidad de uso}
	\end{itemize}
\item \textbf {Exprimir las capacidades informáticas}
	\begin{itemize}
		\item {Falta de conocimiento por parte de los componentes del equipo}
	\end{itemize}
\end{itemize}



\part{Análisis del riesgo}

\subsection*{Nombre del riesgo}			% Hay que rellenar los siguientes campos de cada riesgo
	\begin{itemize}
		\item \textbf {Prioridad: }
		\item \textbf {Probabilidad: }
		\item \textbf {Consecuencia: }
		\item \textbf {Indicios de que se produzca: }
		\item \textbf {Prevencion: }
		\item \textbf {Mitigacion: }
		\item \textbf {Contigencia: }
	\end{itemize}


\section{Defciencias del personal}
%
\subsection*{Baja de algún miembro}			% Hay que rellenar los siguientes campos de cada riesgo
	\begin{itemize}
		\item \textbf {Prioridad: }Alta.
		\item \textbf {Probabilidad: }Frecuente.
		\item \textbf {Consecuencia: }Aumento de la carga de trabajo entre los restantes miembros, disminución de la calidad del producto, retraso en las entregas.
		\item \textbf {Indicios de que se produzca: }
		\item \textbf {Prevencion: }No se puede prever
		\item \textbf {Mitigacion: }
		\item \textbf {Contigencia: }
	\end{itemize}

%
\section{Planificaciones y presupuestos poco realistas}
\subsection*{Retraso en las entregas por mala planificación}			% Hay que rellenar los siguientes campos de cada riesgo
	\begin{itemize}
		\item \textbf {Prioridad: }Alta.
		\item \textbf {Probabilidad: }Frecuente.
		\item \textbf {Consecuencia: }Catastrófica.
		\item \textbf {Indicios de que se produzca: }
		\item \textbf {Prevencion: }
		\item \textbf {Mitigacion: }
		\item \textbf {Contigencia: }
	\end{itemize}

%
\section{Desarrollo de las funciones y propiedades erróneas}
\subsection*{El producto no se ajusta a lo que el cliente necesita por falta de comunicacion}			% Hay que rellenar los siguientes campos de cada riesgo
	\begin{itemize}
		\item \textbf {Prioridad: }Medio-alta.
		\item \textbf {Probabilidad: }Probable.
		\item \textbf {Consecuencia: }Crítica. Habría que rehacer gran parte del proyecto, con todo el coste que esto supone.
		\item \textbf {Indicios de que se produzca: }
		\item \textbf {Prevencion: }Organizar más reuniones con los clientes, para que estos expongan qué es exactamente lo que quieren, y así se pueda realizar el proyecto correctamente. 
		\item \textbf {Mitigacion: }
		\item \textbf {Contigencia: }
	\end{itemize}

%
\section{Continua corriente de cambios en los requisitos}
\subsection*{El cliente no sabe qué espera que haga el producto}		

	\begin{itemize}
		\item \textbf {Prioridad: }Medio-Alta.
		\item \textbf {Probabilidad: }Probable.
		\item \textbf {Consecuencia: }Seria. Supondría tener que estar rediseñando requisitos, reescribiendo código y rehaciendo el producto según le pareciera al cliente.
		\item \textbf {Indicios de que se produzca: }
		\item \textbf {Prevencion: }Desarrollar prototipos, para que sean probados por el cliente y que este sepa qué es lo que espera de su producto.
		\item \textbf {Mitigacion: }
		\item \textbf {Contigencia: }
	\end{itemize}

	% MONITORIZACION DEL RIESGO?

	% Priorizacion del riesgo --> Diapositiva 34
\part{Priorización del riesgo}
	% En la diapositiva 29 hay un modelo para rellenar los campos de cada riesgo




\end{document}

%=======
%>>>>>>> .r30
=======
\pagenumbering{arabic}	

\part{Identificación de los riesgos} %%% PRIMERA PARTE____________________________________________________________________________________

\section{Deficiencias del personal}

	\subsection{Baja de algun miembro del equipo}

	\subsection {Baja del supervisor del proyecto}


\section{Planificaciones y presupuestos poco realistas}

	\subsection{Retraso en las entregas por mala planificacion}

	\subsection{No entregar todo lo acordado en la planificacion por falta de tiempo}

	\subsection{Perdidas insubsanables}

	\subsection{Mayores gastos de lo esperado}

	\subsection{Cierre del proyecto por ser insostenible}


\section{Desarrollo de las funciones y propiedades erróneas}

	\subsection{Las funciones son ineficientes}

	\subsection{Poca calidad de las funciones y propiedades realizadas}

	\subsection{Las distintas partes del proyecto no cumplen con su cometido}

	\subsection{Dificultad para hacer que las distintas funciones del proyecto se coordinen entre ellas}

	\subsection{El producto no se ajusta a lo que el cliente necesita por falta de comunicacion}

	\subsection{El cliente rechaza las funciones que hemos desarrollado}

	\subsection{El cliente no sabe que funciones debe desarrollar el producto}
	Aqui solo escribo para que salga bien el pdf, quitar esto al hacerlo...
	\subsection{El producto no funciona debidamente en la plataforma en que se quiere usar}

	\subsection{El lenguaje no permite realizar todas las funciones}


\section{Desarrollo erróneo de la interfaz de usuario}

	\subsection{La interfaz de usuario es demasiado dificil de construir}

	\subsection{Falta de recursos para el desarrollo de la interfaz}

	\subsection{El cliente considera que la interfaz es dificil de usar}

	\subsection{Al cliente no le resulta atractiva la interfaz de usuario}

	\subsection{El cliente decide cambiar por completo la interfaz de usuario}


\section{Chapado}
	
	\subsection{Disminucion de los funciones debido al alto coste}

	\subsection{Menor calidad del producto para disminuir el gasto}

	\subsection{Abandono del proyecto}


\section{Continua corriente de cambios en los requisitos}
	
	\subsection{El cliente cambia de opinión acerca de lo que debe hacer el proyecto}

	\subsection{El cliente no sabe que espera que haga el producto}

	\subsection{Los distintos clientes aportan visiones muy distintas del producto}
	Aqui solo escribo para que salga bien el pdf, quitar esto al hacerlo...

\section{Deficiencias en componentes proporcionados externamente}
	
	\subsection{Las librerias de java no son eficientes para nuestro proyecto}

	\subsection{Los programas proporcionados son muy dificiles de usar y poco efectivos}

	\subsection{Los recursos son proporcionados demasiado tarde}


\section{Deficiencias en tareas desarrolladas externamente}
	
	\subsection{Poco tiempo para realizar correciones}

	\subsection{Poco tiempo para asimilar los pasos a seguir}


\section{Deficiencias en rendimiento en tiempo real}

	\subsection{Falta de recursos para realizar el proyecto}

	\subsection{Nuestro producto no cumple con los requisitos de rendimiento}

	\subsection{Nuestro producto no garantiza la calidad de uso}


\section{Exprimir las capacidades informáticas}
	\subsection{Falta de conocimiento por parte de los componentes del equipo}


\newpage
\mbox{}
\thispagestyle{empty}						% Hoja en blanco, sin numeros ni nada
\newpage
\setcounter{section}{0}

\part{Análisis, priorización y planificación del riesgo}  %%% SEGUNDA PARTE_______________________________________________________________

\section*{Nombre del riesgo}			% Hay que rellenar los siguientes campos de cada riesgo
	\begin{itemize}				% NO SE SI NUMERAR LOS RIESGOS, YO CREO QUE MEJOR SI, PERO COMO SE QUIERA
		\item \textbf {Prioridad: }
		\item \textbf {Probabilidad: }
		\item \textbf {Consecuencia: }
		\item \textbf {Indicios de que se produzca: }
		\item \textbf {Prevencion: }
		\item \textbf {Mitigacion: }
		\item \textbf {Contigencia: }
	\end{itemize}

	% MONITORIZACION DEL RIESGO?

	% Priorizacion del riesgo --> Diapositiva 34

	% En la diapositiva 29 hay un modelo para rellenar los campos de cada riesgo



\newpage
\mbox{}
\thispagestyle{empty}						% Hoja en blanco, sin numeros ni nada, al final del documento
\newpage
\newpage
\part{Identificación de los riesgo}
\begin{itemize}
\item \textbf {Deficiencias del personal}
	\begin{itemize}
		\item {Baja de algun miembro del equipo}
		\item {Baja del supervisor del proyecto}
	\end{itemize}
\item \textbf {Planificaciones y presupuestos poco realistas}
	\begin{itemize}
		\item {Retraso en las entregas por mala planificacion}
		\item {No entregar todo lo acordado en la planificacion por falta de tiempo}
		\item {Perdidas insubsanables}
		\item {Mayores gastos de lo esperado}
		\item {Cierre del proyecto por ser insostenible}
	\end{itemize}
\item \textbf {Desarrollo de las funciones y propiedades erróneas}
	\begin{itemize}
		\item {Las funciones son ineficientes}
		\item {Poca calidad de las funciones y propiedades realizadas}
		\item {Las distintas partes del proyecto no cumplen con su cometido}
		\item {Dificultad para hacer que las distintas funciones del proyecto se coordinen entre ellas}
		\item {El producto no se ajusta a lo que el cliente necesita por falta de comunicacion}
		\item {El cliente rechaza las funciones que hemos desarrollado}
		\item {El cliente no sabe que funciones debe desarrollar el producto}
		\item {El producto no funciona debidamente en la plataforma en que se quiere usar}
		\item {El lenguaje no permite realizar todas las funciones}
	\end{itemize}
\item \textbf {Desarrollo erróneo del interfaz de usuario}
	\begin{itemize}
		\item {La interfaz de usuario es demasiado dificil de construir}
		\item {Falta de recursos para el desarrollo de la interfaz}
		\item {El cliente considera que la interfaz es dificil de usar}
		\item {Al cliente no le resulta atractiva la interfaz de usuario}
		\item {El cliente decide cambiar por completo la interfaz de usuario}
	\end{itemize}
\item \textbf {Chapado}
	\begin{itemize}
		\item {Disminucion de los funciones debido al alto coste}
		\item {Menor calidad del producto para disminuir el gasto}
		\item {Abandono del proyecto}
	\end{itemize}
\item \textbf {Continua corriente de cambios en los requisitos}
	\begin{itemize}
		\item {El cliente cambia de opinión acerca de lo que debe hacer el proyecto}
		\item {El cliente no sabe que espera que haga el producto}
		\item {Los distintos clientes aportan visiones muy distintas del producto}

	\end{itemize}
\item \textbf {Deficiencias en componentes proporcionados externamente}
	\begin{itemize}
		\item {Las librerias de java no son eficientes para nuestro proyecto}
		\item {Los programas proporcionados son muy dificiles de usar y poco efectivos}
		\item {Los recursos son proporcionados demasiado tarde}

	\end{itemize}
\item \textbf {Deficiencias en tareas desarrolladas externamente}
	\begin{itemize}
		\item {Poco tiempo para realizar correciones}
		\item {Poco tiempo para asimilar los pasos a seguir}
	\end{itemize}
\item \textbf {Deficiencias en rendimiento en tiempo real}
	\begin{itemize}
		\item {Falta de recursos para realizar el proyecto}
		\item {Nuestro producto no cumple con los requisitos de rendimiento}
		\item {Nuestro producto no garantiza la calidad de uso}
	\end{itemize}
\item \textbf {Exprimir las capacidades informáticas}
	\begin{itemize}
		\item {Falta de conocimiento por parte de los componentes del equipo}
	\end{itemize}
\end{itemize}

\newpage
\newpage

\part{Análisis, priorización y planificación del riesgo}
\subsection*{Nombre del riesgo}			% Hay que rellenar los siguientes campos de cada riesgo
	\begin{itemize}
		\item \textbf {Prioridad:}
		\item \textbf {Probabilidad:}
		\item \textbf {Consecuencia:}
		\item \textbf {Indicios de que se produzca:}
		\item \textbf {Prevencion:}
		\item \textbf {Mitigacion:}
		\item \textbf {Contigencia:}
	\end{itemize}

	% MONITORIZACION DEL RIESGO?

	% Priorizacion del riesgo --> Diapositiva 34

	% En la diapositiva 29 hay un modelo para rellenar los campos de cada riesgo



>>>>>>> .r52



<<<<<<< .mine
%<<<<<<< .mine
\pagenumbering{arabic}	

\part{Identificación de los riesgos} %%% PRIMERA PARTE____________________________________________________________________________________

\section{Deficiencias del personal}

	\subsection{Baja de algun miembro del equipo}

	\subsection {Baja del supervisor del proyecto}


\section{Planificaciones y presupuestos poco realistas}

	\subsection{Retraso en las entregas por mala planificacion}

	\subsection{No entregar todo lo acordado en la planificacion por falta de tiempo}

	\subsection{Perdidas insubsanables}

	\subsection{Mayores gastos de lo esperado}

	\subsection{Cierre del proyecto por ser insostenible}


\section{Desarrollo de las funciones y propiedades erróneas}

	\subsection{Las funciones son ineficientes}

	\subsection{Poca calidad de las funciones y propiedades realizadas}

	\subsection{Las distintas partes del proyecto no cumplen con su cometido}

	\subsection{Dificultad para hacer que las distintas funciones del proyecto se coordinen entre ellas}

	\subsection{El producto no se ajusta a lo que el cliente necesita por falta de comunicacion}

	\subsection{El cliente rechaza las funciones que hemos desarrollado}

	\subsection{El cliente no sabe que funciones debe desarrollar el producto}
	Aqui solo escribo para que salga bien el pdf, quitar esto al hacerlo...
	\subsection{El producto no funciona debidamente en la plataforma en que se quiere usar}

	\subsection{El lenguaje no permite realizar todas las funciones}


\section{Desarrollo erróneo de la interfaz de usuario}

	\subsection{La interfaz de usuario es demasiado dificil de construir}

	\subsection{Falta de recursos para el desarrollo de la interfaz}

	\subsection{El cliente considera que la interfaz es dificil de usar}

	\subsection{Al cliente no le resulta atractiva la interfaz de usuario}

	\subsection{El cliente decide cambiar por completo la interfaz de usuario}


\section{Chapado}
	
	\subsection{Disminucion de los funciones debido al alto coste}

	\subsection{Menor calidad del producto para disminuir el gasto}

	\subsection{Abandono del proyecto}


\section{Continua corriente de cambios en los requisitos}
	
	\subsection{El cliente cambia de opinión acerca de lo que debe hacer el proyecto}

	\subsection{El cliente no sabe que espera que haga el producto}

	\subsection{Los distintos clientes aportan visiones muy distintas del producto}
	Aqui solo escribo para que salga bien el pdf, quitar esto al hacerlo...

\section{Deficiencias en componentes proporcionados externamente}
	
	\subsection{Las librerias de java no son eficientes para nuestro proyecto}

	\subsection{Los programas proporcionados son muy dificiles de usar y poco efectivos}

	\subsection{Los recursos son proporcionados demasiado tarde}


\section{Deficiencias en tareas desarrolladas externamente}
	
	\subsection{Poco tiempo para realizar correciones}

	\subsection{Poco tiempo para asimilar los pasos a seguir}


\section{Deficiencias en rendimiento en tiempo real}

	\subsection{Falta de recursos para realizar el proyecto}

	\subsection{Nuestro producto no cumple con los requisitos de rendimiento}

	\subsection{Nuestro producto no garantiza la calidad de uso}


\section{Exprimir las capacidades informáticas}
	\subsection{Falta de conocimiento por parte de los componentes del equipo}
