%Documento que incluye la gestion de riesgos segun el modelo BOEHM

\documentclass[spanish,a4paper,12pt]{report}	% Idioma, tamaño del papel, tamaño letra, documento (book, report, article, letter)

%%% PAQUETES
\usepackage[spanish,activeacute]{babel}				
% Babel: Adapta cosas como la tipografia, la fecha, lo de Chapter al español, y activeacute para apóstrofes (') como abreviaciones de acentos: \'{a}
\usepackage[utf8]{inputenc}					% Codificacion UTF8 (para meter tildes normal: á --> \'{a} )
\usepackage{multicol}						% Escritura en varias columnas
\usepackage{graphics}						% Inclusión de imágenes
\usepackage{graphicx}						% Mas para imagenes
\usepackage{geometry}						% Distribucion de la pagina: margenes, encabezados, tamaño pagina...
\usepackage{fancyhdr}						% Paquete para añadir y modificar encabezados y pies de pagina
\usepackage{hyperref}						% Para hipervínculos, en el indice al menos, GRACIAS A DAVID
%\usepackage{lastpage}						% Ultima pagina para poner, por ejemplo, 3 de 15
%%% PAQUETES MATEMATICOS
\usepackage{amsmath}						% Conjunto de paquetes desarrollados por la Amercian Matematical Society
\usepackage{amssymb}						% Tipografía mathbb y otros símbolos tambien de la AMS
\usepackage{amsthm}						% Paquete AMS theorem, de la AMS
\usepackage{amsfonts}						% Paquete con símbolos y mas, de la AMS
%\usepackage{nicefrac}						% Fracciones bonitas, LO DEJO COMENTADO PORQUE A VECES DA PROBLEMAS AL COMPILAR


%%% DECLARACIONES (sobre la forma de la pagina, encabezado etc.)
\pagenumbering{roman}						
% Para numerar las paginas en numeros romanos hasta que empiece el texto (tambien alph, Alph, roman, Roman...)
\pagestyle{fancy}							% Utiliza el paquete fancyhdr para encabezados y pies de pagina
%\thispagestyle{empty}  						% Para poner UNA pagina sin encabezados ni numero, "plain" CON numero, "fancy" normal
%\lhead{Encabezado a la izquierda}				% Encabezado a la izquierda
\rhead{\bfseries Gestión de riesgos}			%Encabezado a la derecha
\cfoot{\thepage}							% Numero de pagina centrado en el pie
%\cfoot{\thepage\ de \pageref{LastPage}}		% Numero de pagina centrado en el pie asi: n de m
\renewcommand{\headrulewidth}{0.4pt}			% Linea debajo del encabezado
\renewcommand{\footrulewidth}{0.4pt}			% Linea encima del pie de pagina
\renewcommand*{\thesection}{\arabic{section}}	% Hace que no apareca el indice de capitulos y que comience en section, GRACIAS A RUBEN


%%%%% CUERPO %%%%%
\begin{document}


\title{\textbf{\huge{Gestión de \\ riesgos Software}}}

\title{\textbf{\huge{Gestión de \\ 
	riesgos Software}} \\ \vspace{0.3cm}
	\Large{Ingeniería del Software}}

\author{ Jesús Aguirre Pemán \\
	 Enrique Ballesteros Horcajo \\
	 Jaime Dan Porras Rhee \\
	 Ignacio Iker Prado Rujas \\
	 Alejandro Villarín Prieto }
\date{\Today}
\maketitle

%Problemas con SVN por mantenimiento
%Pelea dentro del equipo
%Falta de tiempo por 

\tableofcontents 							%INDICE hipervinculado


\newpage
\mbox{}
\thispagestyle{empty}						% Hoja en blanco, sin numeros ni nada
\newpage
\setcounter{section}{0}
\pagenumbering{arabic}						% Pone el contador de paginas a 1 y ahora en numeros normales

\part{Identificación de los riesgos}
\begin{itemize}
\item \textbf {Deficiencias del personal}
	\begin{itemize}
		\item {Baja temporal de algun miembro del equipo por enfermedad}
		\item {Baja definitiva de algun miembro del equipo por enfermedad}
		\item {Abandono de la asignatura por parte de algun miembro del equipo}
		\item {Abandono del proyecto por algun miembro del equipo}
		\item {Abandono de la carrera por algun miembro del equipo}
		\item {Baja del supervisor del proyecto}
	\end{itemize}
\item \textbf {Planificaciones poco realistas}
	\begin{itemize}
		\item {Retraso en las entregas por mala planificacion}

%<<<<<<< .mine
%=======
		\item {No entregar todo lo acordado en la planificacion por falta de tiempo}
		\item {Retraso en las entregas por mala planificación}

		\item {No entregar todo lo acordado en la planificación por falta de tiempo}
		\item {Perdidas insubsanables}
		\item {Mayores gastos de lo esperado}
%>>>>>>> .r75
		\item {Cierre del proyecto por ser insostenible}
	\end{itemize}
\item \textbf {Desarrollo de las funciones y propiedades erróneas}
	\begin{itemize}
		\item {Las funciones son ineficientes}
		\item {Poca calidad de las funciones y propiedades realizadas}
		\item {Las distintas partes del proyecto no cumplen con su cometido}
		\item {Dificultad para hacer que las distintas funciones del proyecto se coordinen entre ellas}
		\item {El producto no se ajusta a lo que el cliente necesita por falta de comunicacion}
		\item {El cliente rechaza las funciones que hemos desarrollado}
		\item {El cliente no sabe que funciones debe desarrollar el producto}
		\item {El producto no funciona debidamente en la plataforma en que se quiere usar}
		\item {El lenguaje no permite realizar todas las funciones}
	\end{itemize}
\item \textbf {Desarrollo erróneo del interfaz de usuario}
	\begin{itemize}
		\item {La interfaz de usuario es demasiado dificil de construir}
		\item {Falta de recursos para el desarrollo de la interfaz}
		\item {El cliente considera que la interfaz es dificil de usar}
		\item {Al cliente no le resulta atractiva la interfaz de usuario}
		\item {El cliente decide cambiar por completo la interfaz de usuario}
	\end{itemize}
\item \textbf {Chapado}
	\begin{itemize}
		\item {Abandono del proyecto}
	\end{itemize}
\item \textbf {Continua corriente de cambios en los requisitos}
	\begin{itemize}
		\item {El cliente cambia de opinión acerca de lo que debe hacer el proyecto}
		\item {El cliente no sabe que espera que haga el producto}
		\item {Los distintos clientes aportan visiones muy distintas del producto}

	\end{itemize}
\item \textbf {Deficiencias en componentes proporcionados externamente}
	\begin{itemize}
		\item {Las librerias de java no son eficientes para nuestro proyecto}
		\item {Los programas proporcionados son muy dificiles de usar y poco efectivos}
		\item {Los recursos son proporcionados demasiado tarde}

	\end{itemize}
\item \textbf {Deficiencias en tareas desarrolladas externamente}
	\begin{itemize}
		\item {Poco tiempo para realizar correciones}
		\item {Poco tiempo para asimilar los pasos a seguir}
		\item {Poco tiempo para realizar el proyecto}
	\end{itemize}
\item \textbf {Deficiencias en rendimiento en tiempo real}
	\begin{itemize}
		\item {Falta de recursos para realizar el proyecto}
		\item {Nuestro producto no cumple con los requisitos de rendimiento}
		\item {Nuestro producto no garantiza la calidad de uso}
	\end{itemize}
\item \textbf {Exprimir las capacidades informáticas}
	\begin{itemize}
		\item {Falta de conocimiento por parte de los componentes del equipo}
	\end{itemize}
\end{itemize}


\newpage
\mbox{}
\thispagestyle{empty}						% Hoja en blanco, sin numeros ni nada
\newpage

\part{Análisis del riesgo}

\subsection*{Nombre del riesgo}			% Hay que rellenar los siguientes campos de cada riesgo
	\begin{itemize}
		\item \textbf {Prioridad: }
		\item \textbf {Probabilidad: }
		\item \textbf {Consecuencia: }
		\item \textbf {Indicios de que se produzca: }
		\item \textbf {Prevencion: }
		\item \textbf {Mitigacion: }
		\item \textbf {Contigencia: }
	\end{itemize}


\section{Deficiencias del personal}
%

\subsection*{Baja temporal de algun miembro del equipo por enfermedad}
	\begin{itemize}
		\item \textbf {Probabilidad: }Frecuente.
		\item \textbf {Consecuencia: }Crítica. Aumento de la carga de trabajo entre los restantes miembros, disminución de la calidad del producto, retraso en las entregas.
	\end{itemize}

\subsection*{Baja definitiva de algun miembro del equipo por enfermedad}	
	\begin{itemize}
		\item \textbf {Probabilidad: }Improbable
		\item \textbf {Consecuencia: }Catastrófica. Aumento de la carga de trabajo entre los restantes miembros, disminución de la calidad del producto, retraso en las entregas.
	\end{itemize}

\subsection*{Abandono de la asignatura por parte de algun miembro del equipo}	
	\begin{itemize}
		\item \textbf {Probabilidad: }Improbable.
		\item \textbf {Consecuencia: }Catastrófica. Aumento de la carga de trabajo entre los restantes miembros, disminución de la calidad del producto, retraso en las entregas.
	\end{itemize}

\subsection*{Abandono del proyecto por algun miembro del equipo}	
	\begin{itemize}
		\item \textbf {Probabilidad: }Improbable.
		\item \textbf {Consecuencia: }Catastrófica. Aumento de la carga de trabajo entre los restantes miembros, disminución de la calidad del producto, retraso en las entregas.
	\end{itemize}

\subsection*{Abandono de la carrera por algun miembro del equipo}
	\begin{itemize}
		\item \textbf {Probabilidad: }Improbable.
		\item \textbf {Consecuencia: }Catastrófica. Aumento de la carga de trabajo entre los restantes miembros, disminución de la calidad del producto, retraso en las entregas.
	\end{itemize}

\subsection*{Baja del supervisor del proyecto}
	\begin{itemize}
		\item \textbf {Probabilidad: }Remota.
		\item \textbf {Consecuencia: }Seria. Cambio en la organización del proyecto y en su desarrollo.
	\end{itemize}

%
\section{Planificaciones y presupuestos poco realistas}

\subsection*{Retraso en las entregas por mala planificación}
	\begin{itemize}
		\item \textbf {Probabilidad: }Ocasional.
		\item \textbf {Consecuencia: }Seria. Cambio en la planificación del proyecto y empeoramiento de los resultados.
	\end{itemize}

\subsection*{No entregar todo lo acordado en la planificación por falta de tiempo}
	\begin{itemize}
		\item \textbf {Probabilidad: }Ocasional.
		\item \textbf {Consecuencia: }Seria. Cambio en la planificación del proyecto y empeoramiento de los resultados.
	\end{itemize}

\subsection*{Cierre del proyecto por ser insostenible}
	\begin{itemize}
		\item \textbf {Probabilidad: }Improbable.
		\item \textbf {Consecuencia: }Catastrófica. Suspenso en IS.
	\end{itemize}

%
\section{Desarrollo de las funciones y propiedades erróneas}

\subsection*{Las funciones son ineficientes}
	\begin{itemize}
		\item \textbf {Probabilidad: }Probable.
		\item \textbf {Consecuencia: }Menor. Empeoramiento de la calidad del software.
	\end{itemize}

\subsection*{Poca calidad de las funciones y propiedades realizadas}
	\begin{itemize}
		\item \textbf {Probabilidad: }Probable.
		\item \textbf {Consecuencia: }Menor. Empeoramiento de la calidad del software.
	\end{itemize}

\subsection*{Las distintas partes del proyecto no cumplen con su cometido}
	\begin{itemize}
		\item \textbf {Probabilidad: }Remota
		\item \textbf {Consecuencia: }Crítica. Necesidad de volver a desarrollar las partes del proyecto que no funcionan correctamente.
	\end{itemize}

\subsection*{Dificultad para hacer que las distintas funciones del proyecto se coordinen entre ellas}
	\begin{itemize}
		\item \textbf {Probabilidad: }Probable.
		\item \textbf {Consecuencia: }Menor. Aumento del esfuerzo.
	\end{itemize}

\subsection*{El producto no se ajusta a lo que el cliente necesita por falta de comunicacion}
	\begin{itemize}
		\item \textbf {Probabilidad: }Ocasional.
		\item \textbf {Consecuencia: }Crítica. Necesidad de corregir todos los errores que se han cometido.
	\end{itemize}

\subsection*{El cliente rechaza las funciones que hemos desarrollado}
	\begin{itemize}
		\item \textbf {Probabilidad: }Probable.
		\item \textbf {Consecuencia: }Crítica. Necesidad de cambiar enormemente el desarrollo del proyecto.
	\end{itemize}

\subsection*{El cliente no sabe que funciones debe desarrollar el producto}
	\begin{itemize}
		\item \textbf {Probabilidad: }Probable.
		\item \textbf {Consecuencia: }Seria. Aumento del tiempo necesario para desarrollar el producto.
	\end{itemize}

\subsection*{El producto no funciona debidamente en la plataforma en que se quiere usar}
	\begin{itemize}
		\item \textbf {Probabilidad: }Improbable.
		\item \textbf {Consecuencia: }Catastrófica. Es necesario desarrollar de nuevo el producto.
	\end{itemize}

\subsection*{El lenguaje no permite realizar todas las funciones}
	\begin{itemize}
		\item \textbf {Probabilidad: }Remota.
		\item \textbf {Consecuencia: }Seria. Sera necesario buscar soluciones alternativas.
	\end{itemize}


%
\section{Desarrollo erróneo del interfaz de usuario}

\subsection*{La interfaz de usuario es demasiado dificil de construir}
	\begin{itemize}
		\item \textbf {Probabilidad: }Probable.
		\item \textbf {Consecuencia: }Seria. Aumento del esfuerzo, el coste y gran disminución de la calidad.
	\end{itemize}

\subsection*{Falta de recursos para el desarrollo de la interfaz}
	\begin{itemize}
		\item \textbf {Probabilidad: }Remota.
		\item \textbf {Consecuencia: }Crítica. Gran aumento del esfuerzo y el coste, y gran disminución de la calidad.
	\end{itemize}

\subsection*{El cliente considera que la interfaz es dificil de usar}
	\begin{itemize}
		\item \textbf {Probabilidad: }Probable.
		\item \textbf {Consecuencia: }Seria. Volver a realizar la interfaz de usuario.
	\end{itemize}

\subsection*{Al cliente no le resulta atractiva la interfaz de usuario}
	\begin{itemize}
		\item \textbf {Probabilidad: }Probable.
		\item \textbf {Consecuencia: }Seria. Volver a realizar la interfaz de usuario o modificarla en su mayor parte.
	\end{itemize}

\subsection*{El cliente decide cambiar por completo la interfaz de usuario}
	\begin{itemize}
		\item \textbf {Probabilidad: }Ocasional.
		\item \textbf {Consecuencia: }Crítica. Desarrollo de una nueva interfaz de usuario aumentando coste y esfuerzo.
	\end{itemize}

%
\section{Chapado}

\subsection*{Abandono del proyecto}
	\begin{itemize}
		\item \textbf {Probabilidad: }Improbable.
		\item \textbf {Consecuencia: }Catastrófica. Suspenso en IS 	% Se parece mucho a otra, se podria considerar quitarla.
	\end{itemize}

%
\section{Continua corriente de cambios en los requisitos}

\subsection*{El cliente cambia de opinión acerca de lo que debe hacer el proyecto}
	\begin{itemize}
		\item \textbf {Probabilidad: }Improbable.
		\item \textbf {Consecuencia: }Catastrófica. cambio total en el desarrollo del proyecto. Sería necesario empezar de nuevo.
	\end{itemize}

\subsection*{El cliente no sabe que espera que haga el producto}	
	\begin{itemize}
		\item \textbf {Probabilidad: }Probable.

%<<<<<<< .mine
%=======
		\item \textbf {Consecuencia: }Seria. Supondría tener que estar rediseñando requisitos, reescribiendo código y rehaciendo el producto según le pareciera al cliente.
%>>>>>>> .r75
	\end{itemize}

\subsection*{Los distintos clientes aportan visiones muy distintas del producto}	
	\begin{itemize}
		\item \textbf {Probabilidad: }Probable.
		\item \textbf {Consecuencia: }Seria. Es muy dificil avanzar en el proyecto.
	\end{itemize}

%
\section{Deficiencias en componentes proporcionados externamente}

\subsection*{Las librerias de java no son eficientes para nuestro proyecto}	
	\begin{itemize}
		\item \textbf {Probabilidad: }Improbable.
		\item \textbf {Consecuencia: }Crítica. Sería necesario implementar funciones que pueden llegar a ser muy complejas.
	\end{itemize}

\subsection*{Los programas proporcionados son muy dificiles de usar y poco efectivos}	
	\begin{itemize}
		\item \textbf {Probabilidad: }Probable.
		\item \textbf {Consecuencia: }Crítica. Falta de información, aumento del esfuerzo y necesidad de buscar otras alternativas.
	\end{itemize}

\subsection*{Los recursos son proporcionados demasiado tarde}	
	\begin{itemize}
		\item \textbf {Probabilidad: }Ocasional.
		\item \textbf {Consecuencia: }Crítica. Aumento del esfuerzo, necesidad de buscar otras alternativas, disminución de la calidad y posibles retrasos.
	\end{itemize}

%
\section{Deficiencias en tareas desarrolladas externamente}

\subsection*{Poco tiempo para realizar correciones}	
	\begin{itemize}
		\item \textbf {Probabilidad: }Ocasional.
		\item \textbf {Consecuencia: }Crítica. Los errores son corregidos muy tarde y se propagan mucho.
	\end{itemize}

\subsection*{Poco tiempo para asimilar los pasos a seguir}	
	\begin{itemize}
		\item \textbf {Probabilidad: }Probable.
		\item \textbf {Consecuencia: }Crítica. Aumento desmesurado de la dificultad del proyecto.
	\end{itemize}

%
\section{Deficiencias en rendimiento en tiempo real}

\subsection*{Falta de recursos para realizar el proyecto}	
	\begin{itemize}
		\item \textbf {Probabilidad: }Remota.
		\item \textbf {Consecuencia: }Crítica. Empeoramiento de la calidad software.
	\end{itemize}

\subsection*{Nuestro producto no cumple con los requisitos de rendimiento}	
	\begin{itemize}
		\item \textbf {Probabilidad: }Ocasional.
		\item \textbf {Consecuencia: }Crítica. Necesidad de realizar de nuevo el trabajo para que cumpla con los requisitos mínimos.
	\end{itemize}

\subsection*{Nuestro producto no garantiza la calidad de uso}	
	\begin{itemize}
		\item \textbf {Probabilidad: }Ocasional.
		\item \textbf {Consecuencia: }Crítica. Es necesario realizar de nuevo el producto.
	\end{itemize}

%
\section{Exprimir las capacidades informáticas}

\subsection*{Falta de conocimiento por parte de los componentes del equipo}	
	\begin{itemize}
		\item \textbf {Probabilidad: }Frecuente.
		\item \textbf {Consecuencia: }Crítica. Empeoramiento enorme de la calidad y aumento de la dificultad de desarrollo.
	\end{itemize}



	% MONITORIZACION DEL RIESGO?

\newpage
\mbox{}
\thispagestyle{empty}						% Hoja en blanco, sin numeros ni nada
\newpage

	% Priorizacion del riesgo --> Diapositiva 34
\part{Priorización del riesgo}
%<<<<<<< .mine

%
\setcounter{section}{0}
\section{Deficiencias del personal}
	\begin{itemize}
		\item \textbf{Baja temporal de algun miembro del equipo por enfermedad}
			\begin{itemize}
				\item \textbf{Prioridad: }Intolerable.		% Creo que se deberia modificar algo porque debe ser alto y no intolerable.
			\end{itemize}
		
		\item \textbf{Baja definitiva de algun miembro del equipo por enfermedad}	
			\begin{itemize}
				\item \textbf{Prioridad: }Media.
			\end{itemize}
		
		\item \textbf{Abandono de la asignatura por parte de algun miembro del equipo}	
			\begin{itemize}
				\item \textbf{Prioridad: }Media.
			\end{itemize}
		
		\item \textbf{Abandono del proyecto por algun miembro del equipo}	
			\begin{itemize}
				\item \textbf{Prioridad: }Media.
			\end{itemize}
		
		\item \textbf{Abandono de la carrera por algun miembro del equipo}
			\begin{itemize}
				\item \textbf{Prioridad: }Media.
			\end{itemize}
		
		\item \textbf{Baja del supervisor del proyecto}
			\begin{itemize}
				\item \textbf{Prioridad: }Baja.
			\end{itemize}
	\end{itemize}
%
\section{Planificaciones y presupuestos poco realistas}
	\begin{itemize}
		\item \textbf{Retraso en las entregas por mala planificación}
			\begin{itemize}
				\item \textbf{Prioridad: }Media.
			\end{itemize}
		
		\item \textbf{No entregar todo lo acordado en la planificación por falta de tiempo}
			\begin{itemize}
				\item \textbf{Prioridad: }Media.
			\end{itemize}
		
		\item \textbf{Cierre del proyecto por ser insostenible}
			\begin{itemize}
				\item \textbf{Prioridad: }Media.
			\end{itemize}
	\end{itemize}
%
\section{Desarrollo de las funciones y propiedades erróneas}
	\begin{itemize}
		\item \textbf{Las funciones son ineficientes}
			\begin{itemize}
				\item \textbf{Prioridad: }Media.
			\end{itemize}
		
		\item \textbf{Poca calidad de las funciones y propiedades realizadas}
			\begin{itemize}
				\item \textbf{Prioridad: }Media.
			\end{itemize}
		
		\item \textbf{Las distintas partes del proyecto no cumplen con su cometido}
			\begin{itemize}
				\item \textbf{Prioridad: }Media.
			\end{itemize}
		
		\item \textbf{Dificultad para hacer que las distintas funciones del proyecto se coordinen entre ellas}
			\begin{itemize}
				\item \textbf{Prioridad: }Media.
			\end{itemize}
		
		\item \textbf{El producto no se ajusta a lo que el cliente necesita por falta de comunicacion}
			\begin{itemize}
				\item \textbf{Prioridad: }Alta.
			\end{itemize}
		
		\item \textbf{El cliente rechaza las funciones que hemos desarrollado}
			\begin{itemize}
				\item \textbf{Prioridad: }Intolerable.
			\end{itemize}
		
		\item \textbf{El cliente no sabe que funciones debe desarrollar el producto}
			\begin{itemize}
				\item \textbf{Prioridad: }Alta.
			\end{itemize}
		
		\item \textbf{El producto no funciona debidamente en la plataforma en que se quiere usar}
			\begin{itemize}
				\item \textbf{Prioridad: }Media.
			\end{itemize}
		
		\item \textbf{El lenguaje no permite realizar todas las funciones}
			\begin{itemize}
				\item \textbf{Prioridad: }Baja.
			\end{itemize}
	\end{itemize}

%
\section{Desarrollo erróneo del interfaz de usuario}
	\begin{itemize}
		\item \textbf{La interfaz de usuario es demasiado dificil de construir}
			\begin{itemize}
				\item \textbf{Prioridad: }Alta
			\end{itemize}
		
		\item \textbf{Falta de recursos para el desarrollo de la interfaz}
			\begin{itemize}
				\item \textbf{Prioridad: }Media.
			\end{itemize}
		
		\item \textbf{El cliente considera que la interfaz es dificil de usar}
			\begin{itemize}
				\item \textbf{Prioridad: }Alta.
			\end{itemize}
		
		\item \textbf{Al cliente no le resulta atractiva la interfaz de usuario}
			\begin{itemize}
				\item \textbf{Prioridad: }Alta.
			\end{itemize}
		
		\item \textbf{El cliente decide cambiar por completo la interfaz de usuario}
			\begin{itemize}
				\item \textbf{Prioridad: }Alta.
			\end{itemize}
	\end{itemize}
%
\section{Chapado}
	\begin{itemize}
		\item \textbf{Abandono del proyecto}
			\begin{itemize}
				\item \textbf{Prioridad: }Media.			% Se parece mucho a otra, se podria considerar quitarla.
			\end{itemize}
	\end{itemize}
%
\section{Continua corriente de cambios en los requisitos}
	\begin{itemize}
		\item \textbf{El cliente cambia de opinión acerca de lo que debe hacer el proyecto}
			\begin{itemize}
				\item \textbf{Prioridad: }Media.
			\end{itemize}
		
		\item \textbf{El cliente no sabe que espera que haga el producto}	
			\begin{itemize}
				\item \textbf{Prioridad: }Alta.
			\end{itemize}
		
		\item \textbf{Los distintos clientes aportan visiones muy distintas del producto}	
			\begin{itemize}
				\item \textbf{Prioridad: }Alta.
			\end{itemize}
	\end{itemize}
%
\section{Deficiencias en componentes proporcionados externamente}
	\begin{itemize}
		\item \textbf{Las librerias de java no son eficientes para nuestro proyecto}	
			\begin{itemize}
				\item \textbf{Prioridad: }Baja.
			\end{itemize}
		
		\item \textbf{Los programas proporcionados son muy dificiles de usar y poco efectivos}	
			\begin{itemize}
				\item \textbf{Prioridad: }Intolerable.
			\end{itemize}
		
		\item \textbf{Los recursos son proporcionados demasiado tarde}	
			\begin{itemize}
				\item \textbf{Prioridad: }Alta.
			\end{itemize}
	\end{itemize}
%
\section{Deficiencias en tareas desarrolladas externamente}
	\begin{itemize}
		\item \textbf{Poco tiempo para realizar correciones}	
			\begin{itemize}
				\item \textbf{Prioridad: }Alta.
			\end{itemize}
		
		\item \textbf{Poco tiempo para asimilar los pasos a seguir}	
			\begin{itemize}
				\item \textbf{Prioridad: }Intolerable.
			\end{itemize}
	\end{itemize}
%
\section{Deficiencias en rendimiento en tiempo real}
	\begin{itemize}
		\item \textbf{Falta de recursos para realizar el proyecto}	
			\begin{itemize}
				\item \textbf{Prioridad: }Media.
			\end{itemize}
		
		\item \textbf{Nuestro producto no cumple con los requisitos de rendimiento}	
			\begin{itemize}
				\item \textbf{Prioridad: }Alta.
			\end{itemize}
		
		\item \textbf{Nuestro producto no garantiza la calidad de uso}	
			\begin{itemize}
				\item \textbf{Prioridad: }Alta.
			\end{itemize}
	\end{itemize}
%
\section{Exprimir las capacidades informáticas}
	\begin{itemize}
		\item \textbf{Falta de conocimiento por parte de los componentes del equipo}	
			\begin{itemize}
				\item \textbf{Prioridad: }Intolerable.
			\end{itemize}
	\end{itemize}


\part{Gestión del riesgo}
\begin{itemize}
\item \textbf{Nombre del riesgo}			% Hay que rellenar los siguientes campos de cada riesgo
	\begin{itemize}
		\item \textbf {Indicios de que se produzca: }
		\item \textbf {Acción de contingencia: }
		\item \textbf {Prevencion: }
		\item \textbf {Mitigacion: }
	\end{itemize}
\end{itemize}
%=======
%>>>>>>> .r75
	% En la diapositiva 29 hay un modelo para rellenar los campos de cada riesgo


\newpage
\mbox{}
\thispagestyle{empty}						% Hoja en blanco, sin numeros ni nada, al final del documento
\newpage

\end{document}


\subsection*{Baja de algún miembro}			% Hay que rellenar los siguientes campos de cada riesgo
	\begin{itemize}
		\item \textbf {Prioridad: }Alta.
		\item \textbf {Probabilidad: }Frecuente.
		\item \textbf {Consecuencia: }Crítica. Aumento de la carga de trabajo entre los restantes miembros, disminución de la calidad del producto, retraso en las entregas.
		\item \textbf {Indicios de que se produzca: }
		\item \textbf {Prevencion: }No se puede prever.
		\item \textbf {Mitigacion: }
		\item \textbf {Contigencia: }
	\end{itemize}

\subsection*{Retraso en las entregas por mala planificación}			% Hay que rellenar los siguientes campos de cada riesgo
	\begin{itemize}
		\item \textbf {Prioridad: }Alta.
		\item \textbf {Probabilidad: }Frecuente.
		\item \textbf {Consecuencia: }Catastrófica.
		\item \textbf {Indicios de que se produzca: }
		\item \textbf {Prevencion: }
		\item \textbf {Mitigacion: }
		\item \textbf {Contigencia: }
	\end{itemize}

\subsection*{El cliente no sabe qué espera que haga el producto}		

	\begin{itemize}
		\item \textbf {Prioridad: }Medio-Alta.
		\item \textbf {Probabilidad: }Probable.
		\item \textbf {Consecuencia: }Seria. Supondría tener que estar rediseñando requisitos, reescribiendo código y rehaciendo el producto según le pareciera al cliente.
		\item \textbf {Indicios de que se produzca: }
		\item \textbf {Prevencion: }Desarrollar prototipos, para que sean probados por el cliente y que este sepa qué es lo que espera de su producto.
		\item \textbf {Mitigacion: }
		\item \textbf {Accion de contingencia: }
	\end{itemize}

\subsection*{El producto no se ajusta a lo que el cliente necesita por falta de comunicacion}			% Hay que rellenar los siguientes campos de cada riesgo
	\begin{itemize}
		\item \textbf {Prioridad: }Medio-alta.
		\item \textbf {Probabilidad: }Probable.
		\item \textbf {Consecuencia: }Crítica. Habría que rehacer gran parte del proyecto, con todo el coste que esto supone.
		\item \textbf {Indicios de que se produzca: }
		\item \textbf {Prevencion: }Organizar más reuniones con los clientes, para que estos expongan qué es exactamente lo que quieren, y así se pueda realizar el proyecto correctamente. 
		\item \textbf {Mitigacion: }
		\item \textbf {Contigencia: }
	\end{itemize}