%Documento que incluye DEFINICIONES, ACRONIMOS Y ABREVIATURAS. Es un GLOSARIO

\documentclass[spanish,a4paper,12pt]{report}	% Idioma, tamaño del papel, tamaño letra, documento (book, report, article, letter)

%%% PAQUETES
\usepackage[spanish,activeacute]{babel}				
% Babel: Adapta cosas como la tipografia, la fecha, lo de Chapter al español, y activeacute para apóstrofes (') como abreviaciones de acentos: \'{a}
\usepackage[utf8]{inputenc}					% Codificacion UTF8 (para meter tildes normal: á --> \'{a} )
\usepackage{multicol}						% Escritura en varias columnas
\usepackage{graphics}						% Inclusión de imágenes
\usepackage{graphicx}						% Mas para imagenes
\usepackage{geometry}						% Distribucion de la pagina: margenes, encabezados, tamaño pagina...
\usepackage{fancyhdr}						% Paquete para añadir y modificar encabezados y pies de pagina
\usepackage{hyperref}						% Para hipervínculos, en el indice al menos, GRACIAS A DAVID
%\usepackage{lastpage}						% Ultima pagina para poner, por ejemplo, 3 de 15
%%% PAQUETES MATEMATICOS
\usepackage{amsmath}						% Conjunto de paquetes desarrollados por la Amercian Matematical Society
\usepackage{amssymb}						% Tipografía mathbb y otros símbolos tambien de la AMS
\usepackage{amsthm}						% Paquete AMS theorem, de la AMS
\usepackage{amsfonts}						% Paquete con símbolos y mas, de la AMS
%\usepackage{nicefrac}						% Fracciones bonitas, LO DEJO COMENTADO PORQUE A VECES DA PROBLEMAS AL COMPILAR


%%% DECLARACIONES (sobre la forma de la pagina, encabezado etc.)
\pagenumbering{roman}						
% Para numerar las paginas en numeros romanos hasta que empiece el texto (tambien alph, Alph, roman, Roman...)
\pagestyle{fancy}							% Utiliza el paquete fancyhdr para encabezados y pies de pagina
%\thispagestyle{empty}  						% Para poner UNA pagina sin encabezados ni numero, "plain" CON numero, "fancy" normal
\lhead{}									% Encabezado a la izquierda
\rhead{\bfseries Definiciones, acrónimos y abreviaturas}			%Encabezado a la derecha
\cfoot{\thepage}							% Numero de pagina centrado en el pie
%\cfoot{\thepage\ de \pageref{LastPage}}		% Numero de pagina centrado en el pie asi: n de m
\renewcommand{\headrulewidth}{0.4pt}			% Linea debajo del encabezado
\renewcommand{\footrulewidth}{0.4pt}			% Linea encima del pie de pagina
\renewcommand*{\thesection}{\arabic{section}}	% Hace que no apareca el indice de capitulos y que comience en section, GRACIAS A RUBEN
\renewcommand*{\thesubsection}{\arabic{subsection}}
\renewcommand*{\thesubsubsection}{\arabic{subsubsection}}

%%%%% CUERPO %%%%%
\begin{document}

\title{\textbf{\huge{Definiciones, acrónimos \\ 
	y abreviaturas}} \\ \vspace{0.3cm}
	\Large{Ingeniería del Software}}
\author{ Jesús Aguirre Pemán \\
	 Enrique Ballesteros Horcajo \\
	 Jaime Dan Porras Rhee \\
	 Ignacio Iker Prado Rujas \\
	 Alejandro Villarín Prieto }
\date{\Today}
\maketitle

\newpage
\mbox{}
\thispagestyle{empty}						% Hoja en blanco, sin numeros ni nada
\newpage


\tableofcontents 							%INDICE hipervinculado

\newpage
\mbox{}
\thispagestyle{empty}						% Hoja en blanco, sin numeros ni nada
\newpage	

\subsection{Maître:} Jefe de comedor en un restaurante. 
\subsection{Comanda:} Pedido que se hace al camarero en un restaurante.
\subsection{Log in:} Anglicismo que sustituye en el mundo informático a la palabra española “Registrarse”.
\subsection{Comensal:} Cada una de las personas que comen en una misma mesa.
\subsection{Pitanza:} Ración de comida que se distribuye a quienes viven en comunidad.
\subsection{Chef:} Jefe de cocina, en especial de un restaurante.
\subsection{PDF:} Acrónimo de Portable Document Format (en castellano, Formato de Documento Portátil). Formato de almacenamiento de documentos digitales independiente de plataformas de software o hardware.
\subsection{C.V.:}Acrónimo de Currículum Vítae. Relación de los títulos, honores, cargos, trabajos realizados, datos biográficos, etc., que califican a una persona.
\subsection{Tablet:} Computadora portátil de mayor tamaño que un teléfono inteligente o una PDA, integrado en una pantalla táctil con la que se interactúa primariamente con los dedos o una pluma stylus, sin necesidad de teclado físico ni ratón.
\subsection{Hardware:} Conjunto de los componentes que integran la parte material de una computadora.
\subsection{Software:} Conjunto de programas, instrucciones y reglas informáticas para ejecutar ciertas tareas en una computadora.
\subsection{Windows:}Familia de sistemas operativos desarrollados y vendidos por Microsoft. Las versiones utilizadas en nuestra aplicación serán Windows 7 y Windows Phone 8.
\subsection{Android:} Sistema operativo basado en Linux, diseñado principalmente para móviles con pantalla táctil como teléfonos inteligentes o tabletas inicialmente desarrollados por Android Inc, empresa propiedad de Google Inc. Las versiones utilizadas serán desde Gingerbread (2.3) a Jellybean (4.2).
\subsection{iOS:} Sistema operativo móvil de la empresa Apple Inc. La versión utilizada para nuestra aplicacion será iOS6.
\subsection{Java:} Lenguaje de programación para aplicaciones software independiente de la plataforma.
\subsection{Wi-Fi:} Mecanismo de conexión de dispositivos electrónicos de forma inalámbrica.
\subsection{Procesador:} Programa o aparato para el procesamiento de datos.
\subsection{Base de datos:} Conjunto de datos pertenecientes a un mismo contexto y almacenados sistemáticamente para su posterior uso.
\subsection{Benchmark:} Técnica utilizada para medir el rendimiento de un sistema o componente del mismo.


\newpage
\mbox{}
\thispagestyle{empty}						% Hoja en blanco, sin numeros ni nada, al final del documento
\newpage

\pagenumbering{arabic}						% Pone el contador de paginas a 1 y ahora en numeros normales

\end{document}