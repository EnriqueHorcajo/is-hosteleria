%Documento con el plan de proyecto, estimacion etc. segun PRESSMAN

\documentclass[spanish,a4paper,12pt]{report}	% Idioma, tamaño del papel, tamaño letra, documento (book, report, article, letter)

%%% PAQUETES
\usepackage[spanish,activeacute]{babel}				
% Babel: Adapta cosas como la tipografia, la fecha, lo de Chapter al español, y activeacute para apóstrofes (') como abreviaciones de acentos: \'{a}
\usepackage[utf8]{inputenc}					% Codificacion UTF8 (para meter tildes normal: á --> \'{a} )
\usepackage{multicol}						% Escritura en varias columnas
\usepackage{graphics}						% Inclusión de imágenes
\usepackage{graphicx}						% Mas para imagenes
\usepackage{geometry}						% Distribucion de la pagina: margenes, encabezados, tamaño pagina...
\usepackage{fancyhdr}						% Paquete para añadir y modificar encabezados y pies de pagina
\usepackage{hyperref}						% Para hipervínculos, en el indice al menos, GRACIAS A DAVID
%\usepackage{lastpage}						% Ultima pagina para poner, por ejemplo, 3 de 15
%%% PAQUETES MATEMATICOS
\usepackage{amsmath}						% Conjunto de paquetes desarrollados por la Amercian Matematical Society
\usepackage{amssymb}						% Tipografía mathbb y otros símbolos tambien de la AMS
\usepackage{amsthm}						% Paquete AMS theorem, de la AMS
\usepackage{amsfonts}						% Paquete con símbolos y mas, de la AMS
%\usepackage{nicefrac}						% Fracciones bonitas, LO DEJO COMENTADO PORQUE A VECES DA PROBLEMAS AL COMPILAR


%%% DECLARACIONES (sobre la forma de la pagina, encabezado etc.)
\pagenumbering{roman}						
% Para numerar las paginas en numeros romanos hasta que empiece el texto (tambien alph, Alph, roman, Roman...)
\pagestyle{fancy}							% Utiliza el paquete fancyhdr para encabezados y pies de pagina
%\thispagestyle{empty}  						% Para poner UNA pagina sin encabezados ni numero, "plain" CON numero, "fancy" normal
%\lhead{Encabezado a la izquierda}				% Encabezado a la izquierda
\rhead{\bfseries Plan de proyecto}				%Encabezado a la derecha
\cfoot{\thepage}							% Numero de pagina centrado en el pie
%\cfoot{\thepage\ de \pageref{LastPage}}		% Numero de pagina centrado en el pie asi: n de m
\renewcommand{\headrulewidth}{0.4pt}			% Linea debajo del encabezado
\renewcommand{\footrulewidth}{0.4pt}			% Linea encima del pie de pagina
\renewcommand*{\thesection}{\arabic{section}}	% Hace que no apareca el indice de capitulos y que comience en section, GRACIAS A RUBEN


%%%%% CUERPO %%%%%
\begin{document}

\title{\textbf{\huge{Plan de proyecto}} \\ \vspace{0.3cm}
	\Large{Ingeniería del Software}}
\author{ Jesús Aguirre Pemán \\
	 Enrique Ballesteros Horcajo \\
	 Jaime Dan Porras Rhee \\
	 Ignacio Iker Prado Rujas \\
	 Alejandro Villarín Prieto }
\date{\Today}
\maketitle

\newpage
\mbox{}
\thispagestyle{empty}						% Hoja en blanco, sin numeros ni nada
\newpage


\tableofcontents 							%INDICE hipervinculado

\newpage
\mbox{}
\thispagestyle{empty}						% Hoja en blanco, sin numeros ni nada
\newpage

\pagenumbering{arabic}						% Pone el contador de paginas a 1 y ahora en numeros normales


%INTRODUCCIÓN HECHA POR PERELMANIYA, petaqueo total
\part{Introducción}

	\section{Propósito del plan}

	\section{Ámbito del proyecto y objetivos}

		\subsection{Declaración del ámbito}

		\subsection{Funciones principales}

		\subsection{Aspectos de rendimiento}

		\subsection{Restricciones y técnicas de gestión}

	\section{Modelo de proceso}

\newpage
\mbox{}
\thispagestyle{empty}						% Hoja en blanco, sin numeros ni nada
\newpage
\setcounter{section}{0}

%IKER
\part{Estimaciones del proyecto}

La estimación en un proyecto software es una de las partes indispensables dentro de la planificación, aunque es complicada y requiere experiencia. Por otro lado, es claro que nunca podrá ser definitiva y perfecta, pues el desarrollo de software sufre continuos cambios a lo largo de su vida. 

De todos modos, una buena estimación resulta beneficiosa, pues puede ahorrar bastante tiempo, que es esencial en el proyecto. Además, proporciona un marco de trabajo, para fijar fechas, costes y recursos, y cuándo estos se van a utilizar.

	\section{Datos históricos}
	No se dispone de ésta información, tratándose de un proyecto como el nuestro, académico, que se podría considerar de reingeniería. 

	\section{Técnicas de estimación}
	Existen varias, pero la que vamos a utilizar se encuadra dentro de las técnicas de descomposición basadas en el problema, y nace del estudio del tamaño del software en base a su funcionalidad (Puntos de Función o PF). No es baladí subrayar esto último: los PF miden funcionalidad, no complejidad. 

	%Bla bla bla

	Para más detalle, se dispone del documento adjunto \texttt{Estimación del proyecto Software}, donde se estudia en profundidad el número de puntos de función y su origen.

	\section{Estimaciones de esfuerzo, coste y duración}
	%ESFUERZO: Personas por mes; COSTE: €; DURACION: Tiempo, supongo que en funcion del nº personas (5)
	Como conclusión del apartado anterior, podemos obtener una estimación en esfuerzo, dinero y tiempo para el producto.

	%Bla bla bla

	De nuevo, esto tan sólo es una visión global. Este contenido está ampliado en el documento anexo \texttt{Estimación del proyecto Software}.


\newpage
\mbox{}
\thispagestyle{empty}						% Hoja en blanco, sin numeros ni nada
\newpage
\setcounter{section}{0}

%VILLARIN & JAIME
\part{Estrategia de gestión del riesgo}

	\section{Análisis del riesgo}

	\section{Estudio de los riesgos}

	\section{Plan de gestión del riesgo}

\newpage
\mbox{}
\thispagestyle{empty}						% Hoja en blanco, sin numeros ni nada
\newpage
\setcounter{section}{0}

%KIKE
\part{Planificación temporal}

	\section{Estructura de descomposición del trabajo, Planificación temporal}

	\section{Gráfico de  Gantt}

	\section{Red de tareas}

	\section{Tabla de uso de recursos}

\newpage
\mbox{}
\thispagestyle{empty}						% Hoja en blanco, sin numeros ni nada
\newpage
\setcounter{section}{0}

%JAIME

\part{Recursos del proyecto}

	\section{Personal}

	\section{Hardware y software}

	\section{Lista de recursos}

\newpage
\mbox{}
\thispagestyle{empty}						% Hoja en blanco, sin numeros ni nada
\newpage
\setcounter{section}{0}

%JESUS

\part{Organización del personal}

	\section{Estructura de equipo}

	\section{Informes de gestión}

\newpage
\mbox{}
\thispagestyle{empty}						% Hoja en blanco, sin numeros ni nada
\newpage
\setcounter{section}{0}

%KIKE & JESUS
%ESTO PARA LA PROXIMA ENTREGA PERO NO SE QUITA, SE QUEDA
%HABRIA QUE DECIRLO AUNQUE SEA
\part{Mecanismos de seguimiento y control}

	\section{Garantía de calidad y control}

	\section{Gestión y control de cambios}

\newpage
\mbox{}
\thispagestyle{empty}						% Hoja en blanco, sin numeros ni nada
\newpage
\setcounter{section}{0}
\part{Apéndices}


\newpage
\mbox{}
\thispagestyle{empty}						% Hoja en blanco, sin numeros ni nada, al final del documento
\newpage

\end{document}

%ESQUEMA DEL PLAN DE PROYECTO SEGUN PRESSMAN: SI TENÉIS DUDAS VIENE BASTANTE EXPLICADO EN EL LIBRO DE PRESSMAN
%LA DISTRIBUCION NO ES DEFINITIVA, EN FUNCION DE LO LARGA QUE SEA CADA PARTE LOS QUE ACABEN ANTES AYUDAN AL RESTO

%JESUS
1. Introducción
	1.1 Propósito del plan
	1.2 Ámbito del proyecto y objetivos
		1.2.1 Declaración del ámbito
		1.2.2 Funciones principales
		1.2.3 Aspectos de rendimiento
		1.2.4 Restricciones y técnicas de gestión
	1.3 Modelo de proceso

%IKER
2. Estimaciones del proyecto
	2.1 Datos históricos
	2.2 Técnicas de estimación
	2.3 Estimaciones de esfuerzo, coste y duración

%VILLARIN & JAIME
3. Estrategia de gestión del riesgo 
	3.1 Análisis del riesgo
	3.2 Estudio de los riesgos
	3.3 Plan de gestión del riesgo

%KIKE
4. Planificación temporal
	4.1 Estructura de descomposición del trabajo/Planificación temporal
	4.2 Gráfico de  Gantt
	4.3 Red de tareas
	4.4 Tabla de uso de recursos

%JAIME
5. Recursos del proyecto 
	5.1 Personal
	5.2 Hardware y software
	5.3 Lista de recursos

%JESUS
6. Organización del personal
	6.1 Estructura de equipo (si procede)
	6.2 Informes de gestión
%KIKE & JESUS
7. Mecanismos de seguimiento y control
	7.1 Garantía de calidad y control
	7.2 Gestión y control de cambios 

8. Apéndices