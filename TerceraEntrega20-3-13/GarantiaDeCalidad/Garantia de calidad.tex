%


\documentclass[spanish,a4paper,12pt]{report}	% Idioma, tamaño del papel, tamaño letra, documento (book, report, article, letter)

%%% PAQUETES
\usepackage[spanish,activeacute]{babel}				
% Babel: Adapta cosas como la tipografia, la fecha, lo de Chapter al español, y activeacute para apóstrofes (') como abreviaciones de acentos: \'{a}
\usepackage[utf8]{inputenc}					% Codificacion UTF8 (para meter tildes normal: á --> \'{a} )
\usepackage{multicol}						% Escritura en varias columnas
\usepackage{graphics}						% Inclusión de imágenes
\usepackage{graphicx}						% Mas para imagenes
\usepackage{geometry}						% Distribucion de la pagina: margenes, encabezados, tamaño pagina...
\usepackage{fancyhdr}						% Paquete para añadir y modificar encabezados y pies de pagina
\usepackage{hyperref}						% Para hipervínculos, en el indice al menos, GRACIAS A DAVID
%\usepackage{lastpage}						% Ultima pagina para poner, por ejemplo, 3 de 15
%%% PAQUETES MATEMATICOS
\usepackage{amsmath}						% Conjunto de paquetes desarrollados por la Amercian Matematical Society
\usepackage{amssymb}						% Tipografía mathbb y otros símbolos tambien de la AMS
\usepackage{amsthm}						% Paquete AMS theorem, de la AMS
\usepackage{amsfonts}						% Paquete con símbolos y mas, de la AMS
%\usepackage{nicefrac}						% Fracciones bonitas, LO DEJO COMENTADO PORQUE A VECES DA PROBLEMAS AL COMPILAR


%%% DECLARACIONES (sobre la forma de la pagina, encabezado etc.)
\pagenumbering{roman}						
% Para numerar las paginas en numeros romanos hasta que empiece el texto (tambien alph, Alph, roman, Roman...)
\pagestyle{fancy}							% Utiliza el paquete fancyhdr para encabezados y pies de pagina
%\thispagestyle{empty}  						% Para poner UNA pagina sin encabezados ni numero, "plain" CON numero, "fancy" normal
%\lhead{Encabezado a la izquierda}				% Encabezado a la izquierda
\rhead{\bfseries Garantía de calidad}			%Encabezado a la derecha
\cfoot{\thepage}							% Numero de pagina centrado en el pie
%\cfoot{\thepage\ de \pageref{LastPage}}		% Numero de pagina centrado en el pie asi: n de m
\renewcommand{\headrulewidth}{0.4pt}			% Linea debajo del encabezado
\renewcommand{\footrulewidth}{0.4pt}			% Linea encima del pie de pagina
\renewcommand*{\thesection}{\arabic{section}}	% Hace que no apareca el indice de capitulos y que comience en section, GRACIAS A RUBEN


%%%%% CUERPO %%%%%
\begin{document}

\title{\textbf{\huge{Garantía de \\ 
	calidad del Software}} \\ \vspace{0.3cm}
	\Large{Ingeniería del Software}}
\author{ Jesús Aguirre Pemán \\
	 Enrique Ballesteros Horcajo \\
	 Jaime Dan Porras Rhee \\
	 Ignacio Iker Prado Rujas \\
	 Alejandro Villarín Prieto }
\date{\Today}
\maketitle

\newpage
\mbox{}
\thispagestyle{empty}						% Hoja en blanco, sin numeros ni nada
\newpage


\tableofcontents 							%INDICE hipervinvulado

\newpage
\mbox{}
\thispagestyle{empty}						% Hoja en blanco, sin numeros ni nada
\newpage

\pagenumbering{arabic}						% Pone el contador de paginas a 1 y ahora en numeros normales


%INTRODUCCIÓN HECHA POR KIKE, petaqueo total
\chapter{Propósito del plan de garantía de calidad}
	El propósito de este plan es servir como guía de las actividades destinadas a garantizar que 
	kike-hostelería cumple los requisitos especificados en la documentación.
	(-Delinea el propósito específico y el alcance del plan
	SQA.
	- Lista los nombres de los elementos software
	cubiertos por el plan SQA y el uso de dichos
	elementos.		
	- Determina la porción del ciclo de vida cubierta por
	el plan para cada elemento software.)
\chapter{Documentos de referencia}
	(- Proporciona una lista completa de cualquier
	documento referenciado en el plan o utilizado en su
	elaboración.)
\chapter{Gestión}
	(Tareas relacionadas con la SQA.
	- Idealmente redactado en formato IEEE Std. 1058-
	1998, IEEE Standard for Software Project
	Management Plans
	- Describe la estructura organizativa que influye y
	controla la calidad del software.
	- Identifica roles y responsabilidades dentro del plan
	SQA.
	- Identifica a los responsables de preparar y mantener
	el plan SQA.
	- Identifica las tareas asociadas al proceso SQA.
	- Identifica la relación entre estas tareas y las de la
	planificación temporal)
\chapter{Documentación}
	(- Define toda la documentación que se va a generar
	durante el proceso de desarrollo.
	- Lista los documentos que serán revisados o
	auditados, así como los criterios de revisión.)
\chapter{Estándares, prácticas convenciones y métricas}
	(- Esta sección es un poco miscelánea en SQA.)
\chapter{Revisiones del software}
	(-Determina las revisiones del software.
	- Define los tipos de revisión.
	- Define los productos a revisar.)
\chapter{Prueba}
	(- Identifica todas las pruebas no incluidas en el plan
	de verificación y validación.)
\chapter{Informe de problemas y acción correctiva}
	(- Describe las prácticas y procedimientos de informe,
	seguimiento y resolución de problemas, tanto a nivel	
	producto como proceso.
	- Determina las responsabilidades organizativas
	relativas a su implementación.)
\chapter{Herramientas, técnicas y metodologías}
	(- Herramientas, técnicas y metodologías utilizadas
	para soportar el proceso de SQA.)
\chapter{Control de medios}
	(- Determina los métodos para:
	- Identificar el medio físico de cada producto software.
	- Protegerlo de daños durante el proceso.)
\chapter{Control de proveedor}
	(- Determina las técnicas para garantizar que el
	software proporcionado por proveedores externos
	cumple sus requisitos.
	- También es aplicable a código heredado.)
\chapter{Colección de registros, mantenimiento y conservación}
	(- Identifica la documentación SQA que no se debe
	tirar tras acabar el proceso.
	- Determina los métodos y medios para ensamblar,
	archivar, salvaguardar y mantener la documentación.
	- Fija el periodo de conservación de la información.)
\chapter{Formación}
	(- Identifica las actividades de formación necesarias
	para satisfacer las necesidades del plan SQA.)
\chapter{Gestión del riesgo}
	Está hecho en un documento aparte.
\chapter{Glosario}
	(- Términos específicos del plan SQA.)
\chapter{Procedimiento de cambio e historia del plan SQA}
	(- Procedimientos de modificación del plan SQA.
	- Procedimientos de mantenimiento del historial de
	cambios.
	- Historial de cambios.)
	


\newpage
\mbox{}
\thispagestyle{empty}						% Hoja en blanco, sin numeros ni nada al final del documento
\newpage

\end{document}