%


\documentclass[spanish,a4paper,12pt]{report}	% Idioma, tamaño del papel, tamaño letra, documento (book, report, article, letter)

%%% PAQUETES
\usepackage[spanish,activeacute]{babel}				
% Babel: Adapta cosas como la tipografia, la fecha, lo de Chapter al español, y activeacute para apóstrofes (') como abreviaciones de acentos: \'{a}
\usepackage[utf8]{inputenc}					% Codificacion UTF8 (para meter tildes normal: á --> \'{a} )
\usepackage{multicol}						% Escritura en varias columnas
\usepackage{graphics}						% Inclusión de imágenes
\usepackage{graphicx}						% Mas para imagenes
\usepackage{geometry}						% Distribucion de la pagina: margenes, encabezados, tamaño pagina...
\usepackage{fancyhdr}						% Paquete para añadir y modificar encabezados y pies de pagina
\usepackage{hyperref}						% Para hipervínculos, en el indice al menos, GRACIAS A DAVID
%\usepackage{lastpage}						% Ultima pagina para poner, por ejemplo, 3 de 15
%%% PAQUETES MATEMATICOS
\usepackage{amsmath}						% Conjunto de paquetes desarrollados por la Amercian Matematical Society
\usepackage{amssymb}						% Tipografía mathbb y otros símbolos tambien de la AMS
\usepackage{amsthm}						% Paquete AMS theorem, de la AMS
\usepackage{amsfonts}						% Paquete con símbolos y mas, de la AMS
%\usepackage{nicefrac}						% Fracciones bonitas, LO DEJO COMENTADO PORQUE A VECES DA PROBLEMAS AL COMPILAR


%%% DECLARACIONES (sobre la forma de la pagina, encabezado etc.)
\pagenumbering{roman}						
% Para numerar las paginas en numeros romanos hasta que empiece el texto (tambien alph, Alph, roman, Roman...)
\pagestyle{fancy}							% Utiliza el paquete fancyhdr para encabezados y pies de pagina
%\thispagestyle{empty}  						% Para poner UNA pagina sin encabezados ni numero, "plain" CON numero, "fancy" normal
%\lhead{Encabezado a la izquierda}				% Encabezado a la izquierda
\rhead{\bfseries Gestión de la configuración}		%Encabezado a la derecha
\cfoot{\thepage}							% Numero de pagina centrado en el pie
%\cfoot{\thepage\ de \pageref{LastPage}}		% Numero de pagina centrado en el pie asi: n de m
\renewcommand{\headrulewidth}{0.4pt}			% Linea debajo del encabezado
\renewcommand{\footrulewidth}{0.4pt}			% Linea encima del pie de pagina
\renewcommand*{\thesection}{\arabic{section}}	% Hace que no apareca el indice de capitulos y que comience en section, GRACIAS A RUBEN


%%%%% CUERPO %%%%%
\begin{document}

\title{\textbf{\huge{Gestión de la \\ 
	configuración Software}} \\ \vspace{0.3cm}
	\Large{Ingeniería del Software}}
\author{ Jesús Aguirre Pemán \\
	 Enrique Ballesteros Horcajo \\
	 Jaime Dan Porras Rhee \\
	 Ignacio Iker Prado Rujas \\
	 Alejandro Villarín Prieto }
\date{\Today}
\maketitle

\newpage
\mbox{}
\thispagestyle{empty}						% Hoja en blanco, sin numeros ni nada
\newpage


\tableofcontents 							%INDICE hipervinvulado

\newpage
\mbox{}
\thispagestyle{empty}						% Hoja en blanco, sin numeros ni nada
\newpage

\pagenumbering{arabic}						% Pone el contador de paginas a 1 y ahora en numeros normales

%%% PARTE 1: INTRODUCCIÓN______________________________________________________________________________________________________
\part{Introducción}
	\section{Propósito}
	\section{Alcance}
	%\section{Definición de términos clave} NO SE PONE -> GLOSARIO
	\section{Referencias}

\newpage
\mbox{}
\thispagestyle{empty}						% Hoja en blanco, sin numeros ni nada
\newpage

\setcounter{section}{0}

%%% PARTE 2: GESTIÓN DE LA GCS__________________________________________________________________________________________________
\part{Gestión de la GCS }
	\section{Organización}
	\section{Responsabilidades GCS}
	\section{Políticas, directivas y procedimientos aplicables}

\newpage
\mbox{}
\thispagestyle{empty}						% Hoja en blanco, sin numeros ni nada
\newpage

\setcounter{section}{0}

%%% PARTE 3: ACTIVIDADES DE LA GCS_____________________________________________________________________________________________
\part{Actividades de la GCS}
	\section{Identificación de la configuración}
		\subsection{Identificación de ECSs}
		\subsection{Nombrado de ECSs}
		\subsection{Adquisición de ECSs}
	\section{Contabilidad de estado de configuración} % ¿Alguien tiene idea de que coño es esto?


\newpage
\mbox{}
\thispagestyle{empty}						% Hoja en blanco, sin numeros ni nada
\newpage

\setcounter{section}{0}

%%% PARTE 5: RECURSOS DE LA GCS_______________________________________________________________________________________________
\part{Recursos de la GCS}

\newpage
\mbox{}
\thispagestyle{empty}						% Hoja en blanco, sin numeros ni nada al final del documento
\newpage

\end{document}

%ALEX
1. Introducción
	1.1 Propósito
	1.2 Alcance
	1.3 Definición de términos clave 
	1.4 Referencias

%KIKE
2. Gestión de la GCS 
	2.1 Organización
	2.2 Responsabilidades GCS
	2.3 Políticas, directivas y procedimientos aplicables

%JESÚS & IKER  (3.1 Y 3.3)
3. Actividades de la GCS
	3.1 Identificación de la configuración
		3.1.1 Identificación de ECSs 
		3.1.2 Nombrado de ECSs 
		3.1.3 Adquisición de ECSs
	3.2 Control de la configuración 					%ESTE NO HAY QUE HACERLO
		3.2.1 Petición de cambios
		3.2.2 Evaluación de cambios
		3.2.3 Aprobación o desaprobación de cambios 
		3.2.4 Implementación de cambios
	3.3 Contabilidad de estado de configuración 
	3.4 Auditorias y revisiones de la configuración		%ESTE NO HAY QUE HACERLO 
	3.5 Control de interfaz							%ESTE NO HAY QUE HACERLO
	3.6 Control de la subcontratación/compra			%ESTE NO HAY QUE HACERLO


4. Planificaciones de la GCS							%ESTE NO HAY QUE HACERLO

%JAIME
5. Recursos de la GCS


6. Mantenimiento del plan de GCS						%ESTE NO HAY QUE HACERLO
