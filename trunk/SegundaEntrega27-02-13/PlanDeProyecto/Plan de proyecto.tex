%Documento con el plan de proyecto, estimacion etc. segun PRESSMAN

%%%%% PAQUETES %%%%%
\documentclass[spanish,a4paper,12pt]{report}		% Idioma, tamaño del papel, tamaño letra, documento (book, report, article, letter)
\usepackage[spanish,activeacute]{babel}				% Acentuación
\usepackage[utf8]{inputenc}						% Codificacion UTF8 (para tildes)	
\usepackage{amsmath}							% Amercian Matematics Society entorno
\usepackage{amssymb}							% Tipografía mathbb y otros símbolos
\usepackage{amsthm}							% Paquete AMS theorem
\usepackage{amsfonts}							% Paquete con símbolos y mas
\usepackage{multicol}							% Escritura en varias columnas
\usepackage{graphics}							% Inclusión de imágenes
\usepackage{nicefrac}							% Fracciones bonitas
\usepackage{geometry}							% Formato de la pagina
\usepackage{fancyhdr}

%%%%% DECLARACIONES %%%%%
\geometry{a4paper, textwidth=16cm, textheight=20cm}
\pagestyle{fancy}
\lhead{Ingeniería del Software}
\rhead{\bfseries{Plan de proyecto}}
\cfoot{\thepage}
\renewcommand{\headrulewidth}{0.4pt}
\renewcommand{\footrulewidth}{0.4pt}


%%%%% CUERPO %%%%%
\begin{document}

\title{\textbf{\huge{Plan de proyecto}}}
\author{ Jesús Aguirre Pemán \\
	 Enrique Ballesteros Horcajo \\
	 Jaime Dan Porras Rhee \\
	 Ignacio Iker Prado Rujas \\
	 Alejandro Villarín Prieto }
\date{\today}
\maketitle

\newpage
\mbox{}
\thispagestyle{empty}


\tableofcontents 		%INDICE

\newpage
\mbox{}
\thispagestyle{empty}
\newpage





\end{document}

%ESQUEMA DEL PLAN DE PROYECTO SEGUN PRESSMAN
%LA DISTRIBUCION NO ES DEFINITIVA, EN FUNCION DE LO LARGA QUE SEA CADA PARTE LOS QUE ACABEN ANTES AYUDAN AL RESTO

%JESUS
1. Introducción
	1.1 Propósito del plan
	1.2 Ámbito del proyecto y objetivos
	1.2.1 Declaración del ámbito
	1.2.2 Funciones principales
	1.2.3 Aspectos de rendimiento
	1.2.4 Restricciones y técnicas de gestión
	1.3 Modelo de proceso

%IKER
2. Estimaciones del proyecto
	2.1 Datos históricos
	2.2 Técnicas de estimación
	2.3 Estimaciones de esfuerzo, coste y duración

%VILLARIN & JAIME
3. Estrategia de gestión del riesgo 3.1 Análisis del riesgo
	3.2 Estudio de los riesgos
	3.3 Plan de gestión del riesgo

%KIKE
4. Planificación temporal
	4.1 Estructura de descomposición del trabajo/Planificación temporal
	4.2 Gráfico de  Gantt
	4.3 Red de tareas
	4.4 Tabla de uso de recursos

%JAIME
5. Recursos del proyecto 5.1 Personal
	5.2 Hardware y software
	5.3 Lista de recursos

%JESUS
6. Organización del personal
	6.1 Estructura de equipo (si procede)
	6.2 Informes de gestión
%KIKE & JESUS
7. Mecanismos de seguimiento y control
	7.1 Garantía de calidad y control
	7.2 Gestión y control de cambios 

8. Apéndices