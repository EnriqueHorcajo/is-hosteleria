%Documento que incluye la gestion de riesgos segun el modelo BOEHM

\documentclass[spanish,a4paper,12pt]{report}	% Idioma, tamaño del papel, tamaño letra, documento (book, report, article, letter)

%%% PAQUETES
\usepackage[spanish,activeacute]{babel}				
% Babel: Adapta cosas como la tipografia, la fecha, lo de Chapter al español, y activeacute para apóstrofes (') como abreviaciones de acentos: \'{a}
\usepackage[utf8]{inputenc}					% Codificacion UTF8 (para meter tildes normal: á --> \'{a} )
\usepackage{multicol}						% Escritura en varias columnas
\usepackage{graphics}						% Inclusión de imágenes
\usepackage{graphicx}						% Mas para imagenes
\usepackage{geometry}						% Distribucion de la pagina: margenes, encabezados, tamaño pagina...
\usepackage{fancyhdr}						% Paquete para añadir y modificar encabezados y pies de pagina
\usepackage{hyperref}						% Para hipervínculos, en el indice al menos, GRACIAS A DAVID
%\usepackage{lastpage}						% Ultima pagina para poner, por ejemplo, 3 de 15
%%% PAQUETES MATEMATICOS
\usepackage{amsmath}						% Conjunto de paquetes desarrollados por la Amercian Matematical Society
\usepackage{amssymb}						% Tipografía mathbb y otros símbolos tambien de la AMS
\usepackage{amsthm}						% Paquete AMS theorem, de la AMS
\usepackage{amsfonts}						% Paquete con símbolos y mas, de la AMS
%\usepackage{nicefrac}						% Fracciones bonitas, LO DEJO COMENTADO PORQUE A VECES DA PROBLEMAS AL COMPILAR
 

%%% DECLARACIONES (sobre la forma de la pagina, encabezado etc.)
\pagenumbering{roman}						
% Para numerar las paginas en numeros romanos hasta que empiece el texto (tambien alph, Alph, roman, Roman...)
\pagestyle{fancy}							% Utiliza el paquete fancyhdr para encabezados y pies de pagina
%\thispagestyle{empty}  						% Para poner UNA pagina sin encabezados ni numero, "plain" CON numero, "fancy" normal
%\lhead{Encabezado a la izquierda}				% Encabezado a la izquierda
%\rhead{\bfseries Gestión de riesgos}			%Encabezado a la derecha
\cfoot{\thepage}							% Numero de pagina centrado en el pie
%\cfoot{\thepage\ de \pageref{LastPage}}		% Numero de pagina centrado en el pie asi: n de m
\renewcommand{\headrulewidth}{0.4pt}			% Linea debajo del encabezado
\renewcommand{\footrulewidth}{0.4pt}			% Linea encima del pie de pagina
\renewcommand*{\thesection}{\arabic{section}}	% Hace que no apareca el indice de capitulos y que comience en section, GRACIAS A RUBEN


%%%%% CUERPO %%%%%
\begin{document}


\title{\textbf{\huge{Gestión de \\ riesgos Software}}}

\title{\textbf{\huge{Gestión de \\ 
	riesgos Software}} \\ \vspace{0.3cm}
	\Large{Ingeniería del Software}}

\author{ Jesús Aguirre Pemán \\
	 Enrique Ballesteros Horcajo \\
	 Jaime Dan Porras Rhee \\
	 Ignacio Iker Prado Rujas \\
	 Alejandro Villarín Prieto }
\date{\Today}
\maketitle

%Problemas con SVN por mantenimiento
%Pelea dentro del equipo
%Falta de tiempo por 

\tableofcontents 							%INDICE hipervinculado


\newpage
\mbox{}
\thispagestyle{empty}						% Hoja en blanco, sin numeros ni nada
\newpage
\setcounter{section}{0}
\pagenumbering{arabic}						% Pone el contador de paginas a 1 y ahora en numeros normales

\part{Identificación de los riesgos}
En esta parte se identificarán todos los riesgos que pueden afectar a nuestro proyecto.
\begin{itemize}
\item \textbf {Deficiencias del personal}
	\begin{itemize}
		\item {Baja temporal de algún miembro del equipo por enfermedad}
		\item {Baja definitiva de algún miembro del equipo por enfermedad}
		\item {Abandono de la asignatura por parte de algún miembro del equipo}
		\item {Abandono del proyecto por algún miembro del equipo}
		\item {Abandono de la carrera por algún miembro del equipo}
		\item {Baja del supervisor del proyecto}
	\end{itemize}
\item \textbf {Planificaciones poco realistas}
	\begin{itemize}
		\item {Retraso en las entregas por mala planificacion}
		\item {No entregar todo lo acordado en la planificacion por falta de tiempo}

	\end{itemize}
\item \textbf {Desarrollo de las funciones y propiedades erróneas}
	\begin{itemize}
		\item {Las funciones son ineficientes}
		\item {Poca calidad de las funciones y propiedades realizadas}
		\item {Las distintas partes del proyecto no cumplen con su cometido}
		\item {Dificultad para hacer que las distintas funciones del proyecto se coordinen entre ellas}
		\item {El producto no se ajusta a lo que el cliente necesita por falta de comunicación}
		\item {El cliente rechaza las funciones que hemos desarrollado}
		\item {El cliente no sabe que funciones debe desarrollar el producto}
		\item {El producto no funciona debidamente en la plataforma en que se quiere usar}
		\item {El lenguaje no permite realizar todas las funciones}
		\item {Los programas desarrollados son muy difíciles de usar y poco efectivos}	
	\end{itemize}
\item \textbf {Desarrollo erróneo del interfaz de usuario}
	\begin{itemize}
		\item {La interfaz de usuario es demasiado difícil de construir}
		\item {Falta de recursos para el desarrollo de la interfaz}
		\item {El cliente considera que la interfaz es difícil de usar}
		\item {Al cliente no le resulta atractiva la interfaz de usuario}
		\item {El cliente decide cambiar por completo la interfaz de usuario}
	\end{itemize}
\item \textbf {Chapado}
	\begin{itemize}
		\item {Abandono del proyecto}
	\end{itemize}
\item \textbf {Continua corriente de cambios en los requisitos}
	\begin{itemize}
		\item {El cliente cambia de opinión acerca de lo que debe hacer el proyecto}
		\item {El cliente no sabe que espera que haga el producto}
		\item {Los distintos clientes aportan visiones muy distintas del producto}

	\end{itemize}
\item \textbf {Deficiencias en componentes proporcionados externamente}
	\begin{itemize}
		\item {Las librerías de java no son eficientes para nuestro proyecto}
		\item {Los programas proporcionados son muy difíciles de usar y poco efectivos}
		\item {Los recursos son proporcionados demasiado tarde}

	\end{itemize}
\item \textbf {Deficiencias en tareas desarrolladas externamente}
	\begin{itemize}
		\item {Poco tiempo para realizar correciones}
		\item {Poco tiempo para asimilar los pasos a seguir}
		\item {Poco tiempo para realizar el proyecto}
	\end{itemize}
\item \textbf {Deficiencias en rendimiento en tiempo real}
	\begin{itemize}
		\item {Falta de recursos para realizar el proyecto}
		\item {Nuestro producto no cumple con los requisitos de rendimiento}
		\item {Nuestro producto no garantiza la calidad de uso}
	\end{itemize}
\item \textbf {Exprimir las capacidades informáticas}
	\begin{itemize}
		\item {Falta de conocimiento por parte de los componentes del equipo}
	\end{itemize}
\end{itemize}


\newpage
\mbox{}
\thispagestyle{empty}						% Hoja en blanco, sin numeros ni nada
\newpage

\part{Análisis del riesgo}

%		\subsection*{Nombre del riesgo}			% Hay que rellenar los siguientes campos de cada riesgo
%			\begin{itemize}
%				\item \textbf {Prioridad: }
%				\item \textbf {Probabilidad: }
%				\item \textbf {Consecuencia: }
%				\item \textbf {Indicios de que se produzca: }
%				\item \textbf {Prevención: }
%				\item \textbf {Mitigación: }
%				\item \textbf {Contigencia: }
%			\end{itemize}
En esta parte se analizan todos los riesgos identificados, exponiendo cuál es la probabilidad de que se produzcan y 
explicando la consecuencia que tendrían si llegaran a producirse.

\section{Deficiencias del personal}
%

	\subsection*{Baja temporal de algún miembro del equipo por enfermedad}
		\begin{itemize}
			\item \textbf {Probabilidad: }Frecuente.
			\item \textbf {Consecuencia: }Crítica. Aumento de la carga de trabajo entre los
				restantes miembros, disminución de la calidad del producto, retraso en las
				entregas.
		\end{itemize}
	
	\subsection*{Baja definitiva de algún miembro del equipo por enfermedad}	
		\begin{itemize}
			\item \textbf {Probabilidad: }Improbable
			\item \textbf {Consecuencia: }Catastrófica. Aumento de la carga de trabajo entre los restantes miembros, disminución de la calidad del producto, retraso en las entregas.
		\end{itemize}
	
	\subsection*{Abandono de la asignatura por parte de algún miembro del equipo}	
		\begin{itemize}
			\item \textbf {Probabilidad: }Improbable.
			\item \textbf {Consecuencia: }Catastrófica. Aumento de la carga de trabajo entre los restantes miembros, disminución de la calidad del producto, retraso en las entregas.
		\end{itemize}
	
	\subsection*{Abandono del proyecto por algún miembro del equipo}	
		\begin{itemize}
			\item \textbf {Probabilidad: }Improbable.
			\item \textbf {Consecuencia: }Catastrófica. Aumento de la carga de trabajo entre los restantes miembros, disminución de la calidad del producto, retraso en las entregas.
		\end{itemize}
	
	\subsection*{Abandono de la carrera por algún miembro del equipo}
		\begin{itemize}
			\item \textbf {Probabilidad: }Improbable.
			\item \textbf {Consecuencia: }Catastrófica. Aumento de la carga de trabajo entre los restantes miembros, disminución de la calidad del producto, retraso en las entregas.
		\end{itemize}
	
	\subsection*{Baja del supervisor del proyecto}
		\begin{itemize}
			\item \textbf {Probabilidad: }Remota.
			\item \textbf {Consecuencia: }Seria. Cambio en la organización del proyecto y en su desarrollo.
		\end{itemize}

%
\section{Planificaciones y presupuestos poco realistas}

	\subsection*{Retraso en las entregas por mala planificación}
		\begin{itemize}
			\item \textbf {Probabilidad: }Ocasional.
			\item \textbf {Consecuencia: }Crítica. Cambio en la planificación del proyecto y empeoramiento de los resultados.
		\end{itemize}
	
	\subsection*{No entregar todo lo acordado en la planificación por falta de tiempo}
		\begin{itemize}
			\item \textbf {Probabilidad: }Ocasional.
			\item \textbf {Consecuencia: }Crítica. Cambio en la planificación del proyecto y empeoramiento de los resultados.
		\end{itemize}

%
\section{Desarrollo de las funciones y propiedades erróneas}

	\subsection*{Las funciones son ineficientes}
		\begin{itemize}
			\item \textbf {Probabilidad: }Probable.
			\item \textbf {Consecuencia: }Menor. Empeoramiento de la calidad del software.
		\end{itemize}
	
	\subsection*{Poca calidad de las funciones y propiedades realizadas}
		\begin{itemize}
			\item \textbf {Probabilidad: }Probable.
			\item \textbf {Consecuencia: }Menor. Empeoramiento de la calidad del software.
		\end{itemize}
	
	\subsection*{Las distintas partes del proyecto no cumplen con su cometido}
		\begin{itemize}
			\item \textbf {Probabilidad: }Remota
			\item \textbf {Consecuencia: }Crítica. Necesidad de volver a desarrollar las partes del proyecto que no funcionan correctamente.
		\end{itemize}
	
	\subsection*{Dificultad para hacer que las distintas funciones del proyecto se coordinen entre ellas}
		\begin{itemize}
			\item \textbf {Probabilidad: }Probable.
			\item \textbf {Consecuencia: }Menor. Aumento del esfuerzo.
		\end{itemize}
	
	\subsection*{El producto no se ajusta a lo que el cliente necesita por falta de comunicación}
		\begin{itemize}
			\item \textbf {Probabilidad: }Ocasional.
			\item \textbf {Consecuencia: }Crítica.  Habría que rehacer gran parte del proyecto, con todo el coste que esto supone.
		\end{itemize}
	
	\subsection*{El cliente rechaza las funciones que hemos desarrollado}
		\begin{itemize}
			\item \textbf {Probabilidad: }Probable.
			\item \textbf {Consecuencia: }Crítica. Necesidad de cambiar enormemente el desarrollo del proyecto.
		\end{itemize}
	
	\subsection*{El cliente no sabe qué funciones debe desarrollar el producto}
		\begin{itemize}
			\item \textbf {Probabilidad: }Probable.
			\item \textbf {Consecuencia: }Seria. Aumento del tiempo necesario para desarrollar el producto.
		\end{itemize}
	
	\subsection*{El producto no funciona debidamente en la plataforma en que se quiere usar}
		\begin{itemize}
			\item \textbf {Probabilidad: }Improbable.
			\item \textbf {Consecuencia: }Catastrófica. Es necesario desarrollar de nuevo el producto.
		\end{itemize}
	
	\subsection*{El lenguaje no permite realizar todas las funciones}
		\begin{itemize}
			\item \textbf {Probabilidad: }Remota.
			\item \textbf {Consecuencia: }Seria. Será necesario buscar soluciones alternativas.
		\end{itemize}
		
	\subsection*{Los programas desarrollados son muy difíciles de usar y poco efectivos}	
		\begin{itemize}
			\item \textbf {Probabilidad: }Probable.
			\item \textbf {Consecuencia: }Crítica. El cliente rechaza el producto y habría que rehacerlo de nuevo, 
										  con todo el esfuerzo y costes que esto supone.
		\end{itemize}
	


%
\section{Desarrollo erróneo del interfaz de usuario}

	\subsection*{La interfaz de usuario es demasiado difícil de construir}
		\begin{itemize}
			\item \textbf {Probabilidad: }Probable.
			\item \textbf {Consecuencia: }Seria. Aumento del esfuerzo, el coste y gran disminución de la calidad.
		\end{itemize}
	
	\subsection*{Falta de recursos para el desarrollo de la interfaz}
		\begin{itemize}
			\item \textbf {Probabilidad: }Remota.
			\item \textbf {Consecuencia: }Crítica. Gran aumento del esfuerzo y el coste, y gran disminución de la calidad.
		\end{itemize}
	
	\subsection*{El cliente considera que la interfaz es difícil de usar}
		\begin{itemize}
			\item \textbf {Probabilidad: }Probable.
			\item \textbf {Consecuencia: }Seria. Volver a realizar la interfaz de usuario.
		\end{itemize}
	
	\subsection*{Al cliente no le resulta atractiva la interfaz de usuario}
		\begin{itemize}
			\item \textbf {Probabilidad: }Probable.
			\item \textbf {Consecuencia: }Seria. Volver a realizar la interfaz de usuario o modificarla en su mayor parte.
		\end{itemize}
	
	\subsection*{El cliente decide cambiar por completo la interfaz de usuario}
		\begin{itemize}
			\item \textbf {Probabilidad: }Ocasional.
			\item \textbf {Consecuencia: }Crítica. Desarrollo de una nueva interfaz de usuario aumentando coste y esfuerzo.
		\end{itemize}

%
\section{Chapado}

	\subsection*{Abandono del proyecto}
		\begin{itemize}
			\item \textbf {Probabilidad: }Improbable.
			\item \textbf {Consecuencia: }Catastrófica. Suspenso en IS 	% Se parece mucho a otra, se podria considerar quitarla.
		\end{itemize}

%
\section{Continua corriente de cambios en los requisitos}

	\subsection*{El cliente cambia de opinión acerca de lo que debe hacer el proyecto}
		\begin{itemize}
			\item \textbf {Probabilidad: }Improbable.
			\item \textbf {Consecuencia: }Catastrófica. cambio total en el desarrollo del proyecto. Sería necesario empezar de nuevo.
		\end{itemize}
	
	\subsection*{El cliente no sabe que espera que haga el producto}	
		\begin{itemize}
			\item \textbf {Probabilidad: }Probable.
			\item \textbf {Consecuencia: }Seria. Supondría tener que estar rediseñando requisitos, reescribiendo código y rehaciendo el producto según le pareciera al cliente.
		\end{itemize}
	
	\subsection*{Los distintos clientes aportan visiones muy distintas del producto}	
		\begin{itemize}
			\item \textbf {Probabilidad: }Probable.
			\item \textbf {Consecuencia: }Seria. Es muy difícil avanzar en el proyecto.
		\end{itemize}

%
\section{Deficiencias en componentes proporcionados externamente}

	\subsection*{Las librerías de java no son eficientes para nuestro proyecto}	
		\begin{itemize}
			\item \textbf {Probabilidad: }Improbable.
			\item \textbf {Consecuencia: }Crítica. Sería necesario implementar funciones que pueden llegar a ser muy complejas.
		\end{itemize}
	
	\subsection*{Los programas proporcionados son muy difíciles de usar y poco efectivos}	
		\begin{itemize}
			\item \textbf {Probabilidad: }Probable.
			\item \textbf {Consecuencia: }Crítica. Falta de información, aumento del esfuerzo y necesidad de buscar otras alternativas.
		\end{itemize}
	
	\subsection*{Los recursos son proporcionados demasiado tarde}	
		\begin{itemize}
			\item \textbf {Probabilidad: }Ocasional.
			\item \textbf {Consecuencia: }Crítica. Aumento del esfuerzo, necesidad de buscar otras alternativas, disminución de la calidad y posibles retrasos.
		\end{itemize}

%
\section{Deficiencias en tareas desarrolladas externamente}

	\subsection*{Poco tiempo para realizar correciones}	
		\begin{itemize}
			\item \textbf {Probabilidad: }Ocasional.
			\item \textbf {Consecuencia: }Crítica. Los errores son corregidos muy tarde y se propagan mucho.
		\end{itemize}

	\subsection*{Poco tiempo para asimilar los pasos a seguir}	
		\begin{itemize}
			\item \textbf {Probabilidad: }Probable.
			\item \textbf {Consecuencia: }Crítica. Aumento desmesurado de la dificultad del proyecto.
		\end{itemize}
	
	\subsection*{Poco tiempo para realizar el proyecto}	
		\begin{itemize}
			\item \textbf {Probabilidad: }Ocasional.
			\item \textbf {Consecuencia: }Crítica. Disminución de la calidad del producto.
		\end{itemize}

%
\section{Deficiencias en rendimiento en tiempo real}

	\subsection*{Falta de recursos para realizar el proyecto}	
		\begin{itemize}
			\item \textbf {Probabilidad: }Remota.
			\item \textbf {Consecuencia: }Crítica. Empeoramiento de la calidad software.
		\end{itemize}
	
	\subsection*{Nuestro producto no cumple con los requisitos de rendimiento}	
		\begin{itemize}
			\item \textbf {Probabilidad: }Ocasional.
			\item \textbf {Consecuencia: }Crítica. Necesidad de realizar de nuevo el trabajo para que cumpla con los requisitos mínimos.
		\end{itemize}
	
	\subsection*{Nuestro producto no garantiza la calidad de uso}	
		\begin{itemize}
			\item \textbf {Probabilidad: }Ocasional.
			\item \textbf {Consecuencia: }Crítica. Es necesario realizar de nuevo el producto.
		\end{itemize}
		


%
\section{Exprimir las capacidades informáticas}

	\subsection*{Falta de conocimiento por parte de los componentes del equipo}	
		\begin{itemize}
			\item \textbf {Probabilidad: }Frecuente.
			\item \textbf {Consecuencia: }Crítica. Empeoramiento enorme de la calidad, aumento de la dificultad de desarrollo, retraso en las entregas.
		\end{itemize}
	


	% MONITORIZACION DEL RIESGO?

\newpage
\mbox{}
\thispagestyle{empty}						% Hoja en blanco, sin numeros ni nada
\newpage

	% Priorizacion del riesgo --> Diapositiva 34
\part{Priorización del riesgo}
%<<<<<<< .mine

%
\setcounter{section}{0}
A continuación se exponen las prioridades de cada riesgo, calculadas según su probabilidad y consecuencia. En la siguiente
sección serán ordenados según su prioridad.
\section{Deficiencias del personal}
	\begin{itemize}
		\item \textbf{Baja temporal de algún miembro del equipo por enfermedad}
			\begin{itemize}
				\item \textbf{Prioridad: }Alto.		
			\end{itemize}
		
		\item \textbf{Baja definitiva de algún miembro del equipo por enfermedad}	
			\begin{itemize}
				\item \textbf{Prioridad: }Media.
			\end{itemize}
		
		\item \textbf{Abandono de la asignatura por parte de algún miembro del equipo}	
			\begin{itemize}
				\item \textbf{Prioridad: }Media.
			\end{itemize}
		
		\item \textbf{Abandono del proyecto por algún miembro del equipo}	
			\begin{itemize}
				\item \textbf{Prioridad: }Media.
			\end{itemize}
		
		\item \textbf{Abandono de la carrera por algún miembro del equipo}
			\begin{itemize}
				\item \textbf{Prioridad: }Media.
			\end{itemize}
		
		\item \textbf{Baja del supervisor del proyecto}
			\begin{itemize}
				\item \textbf{Prioridad: }Baja.
			\end{itemize}
	\end{itemize}
%
\section{Planificaciones y presupuestos poco realistas}
	\begin{itemize}
		\item \textbf{Retraso en las entregas por mala planificación}%%%
			\begin{itemize}
				\item \textbf{Prioridad: }Alta.
			\end{itemize}
		
		\item \textbf{No entregar todo lo acordado en la planificación por falta de tiempo}%%%
			\begin{itemize}
				\item \textbf{Prioridad: }Alta.
			\end{itemize}
	\end{itemize}
%
\section{Desarrollo de las funciones y propiedades erróneas}
	\begin{itemize}

		\item \textbf{El cliente rechaza las funciones que hemos desarrollado}
			\begin{itemize}
				\item \textbf{Prioridad: }Intolerable.
			\end{itemize}

		\item \textbf{Los programas desarrollados son muy difíciles de usar y poco efectivos}	
			\begin{itemize}
				\item \textbf{Prioridad: }Intolerable.
			\end{itemize}

		\item \textbf{El producto no se ajusta a lo que el cliente necesita por falta de comunicación}%%%
			\begin{itemize}
				\item \textbf{Prioridad: }Alta.
			\end{itemize}
			
		\item \textbf{El cliente no sabe qué funciones debe desarrollar el producto}%%%
			\begin{itemize}
				\item \textbf{Prioridad: }Alta.
			\end{itemize}
		
		\item \textbf{Las funciones son ineficientes}
			\begin{itemize}
				\item \textbf{Prioridad: }Media.
			\end{itemize}
		
		\item \textbf{Poca calidad de las funciones y propiedades realizadas}
			\begin{itemize}
				\item \textbf{Prioridad: }Media.
			\end{itemize}
		
		\item \textbf{Las distintas partes del proyecto no cumplen con su cometido}
			\begin{itemize}
				\item \textbf{Prioridad: }Media.
			\end{itemize}
		
		\item \textbf{Dificultad para hacer que las distintas funciones del proyecto se coordinen entre ellas}
			\begin{itemize}
				\item \textbf{Prioridad: }Media.
			\end{itemize}
		

		\item \textbf{El producto no funciona debidamente en la plataforma en que se quiere usar}
			\begin{itemize}
				\item \textbf{Prioridad: }Media.
			\end{itemize}
		
		\item \textbf{El lenguaje no permite realizar todas las funciones}
			\begin{itemize}
				\item \textbf{Prioridad: }Baja.
			\end{itemize}

	\end{itemize}

%
\section{Desarrollo erróneo del interfaz de usuario}
	\begin{itemize}
		\item \textbf{La interfaz de usuario es demasiado difícil de construir}%%%
			\begin{itemize}
				\item \textbf{Prioridad: }Alta
			\end{itemize}
				
		\item \textbf{El cliente considera que la interfaz es difícil de usar}%%%
			\begin{itemize}
				\item \textbf{Prioridad: }Alta.
			\end{itemize}
		
		\item \textbf{Al cliente no le resulta atractiva la interfaz de usuario}%%%
			\begin{itemize}
				\item \textbf{Prioridad: }Alta.
			\end{itemize}
		
		\item \textbf{El cliente decide cambiar por completo la interfaz de usuario}%%%No se parece un poco con el anterior?
			\begin{itemize}
				\item \textbf{Prioridad: }Alta.
			\end{itemize}

		\item \textbf{Falta de recursos para el desarrollo de la interfaz}
			\begin{itemize}
				\item \textbf{Prioridad: }Media.
			\end{itemize}

	\end{itemize}
%
\section{Chapado}
	\begin{itemize}
		\item \textbf{Abandono del proyecto}
			\begin{itemize}
				\item \textbf{Prioridad: }Media.			% Se parece mucho a otra, se podria considerar quitarla.
			\end{itemize}
	\end{itemize}
%
\section{Continua corriente de cambios en los requisitos}
	\begin{itemize}
		
		\item \textbf{El cliente no sabe qué espera que haga el producto}	%%%
			\begin{itemize}
				\item \textbf{Prioridad: }Alta.
			\end{itemize}
		
		\item \textbf{Los distintos clientes aportan visiones muy distintas del producto}	%%%
			\begin{itemize}
				\item \textbf{Prioridad: }Alta.
			\end{itemize}

		\item \textbf{El cliente cambia de opinión acerca de lo que debe hacer el proyecto}
			\begin{itemize}
				\item \textbf{Prioridad: }Media.
			\end{itemize}
	\end{itemize}
%
\section{Deficiencias en componentes proporcionados externamente}
	\begin{itemize}
		
		\item \textbf{Los programas proporcionados son muy difíciles de usar y poco efectivos}	
			\begin{itemize}
				\item \textbf{Prioridad: }Alta.
			\end{itemize}
		
		\item \textbf{Los recursos son proporcionados demasiado tarde}	%%%
			\begin{itemize}
				\item \textbf{Prioridad: }Alta.
			\end{itemize}

		\item \textbf{Las librerías de java no son eficientes para nuestro proyecto}	
			\begin{itemize}
				\item \textbf{Prioridad: }Baja.
			\end{itemize}
	\end{itemize}
%
\section{Deficiencias en tareas desarrolladas externamente}
	\begin{itemize}
		\item \textbf{Poco tiempo para asimilar los pasos a seguir}	
			\begin{itemize}
				\item \textbf{Prioridad: }Intolerable.
			\end{itemize}

		\item \textbf{Poco tiempo para realizar correciones}	%%%
			\begin{itemize}
				\item \textbf{Prioridad: }Alta.
			\end{itemize}

\item \textbf{Poco tiempo para realizar el proyecto}
			\begin{itemize}
				\item \textbf{Prioridad: }Alta.
			\end{itemize}
		
	\end{itemize}
%
\section{Deficiencias en rendimiento en tiempo real}
	\begin{itemize}
		\item \textbf{Nuestro producto no cumple con los requisitos de rendimiento}	%%%
			\begin{itemize}
				\item \textbf{Prioridad: }Alta.
			\end{itemize}
		
		\item \textbf{Nuestro producto no garantiza la calidad de uso}%%%	
			\begin{itemize}
				\item \textbf{Prioridad: }Alta.
			\end{itemize}

		\item \textbf{Falta de recursos para realizar el proyecto}	
			\begin{itemize}
				\item \textbf{Prioridad: }Media.
			\end{itemize}
	\end{itemize}
%
\section{Exprimir las capacidades informáticas}
	\begin{itemize}
		\item \textbf{Falta de conocimiento por parte de los componentes del equipo}	
			\begin{itemize}
				\item \textbf{Prioridad: }Intolerable.
			\end{itemize}
	\end{itemize}
	
\section{Prioridad de los riesgos}
		En esta sección se ordenan los riesgos según su prioridad y se eligen aquellos riesgos que van a ser tratados.\\ \ \\ 

			%\textbf{Intolerable}\\
			\begin{tabular}{|p{12cm}|}
				\hline
				\textbf{Riesgos con prioridad Intolerable}\\ \hline \hline
				\textbf{1.} Baja temporal de algún miembro del equipo por enfermedad.\\ \hline 
				\textbf{2.} El cliente rechaza las funciones que hemos desarrollado.\\ \hline
				\textbf{3.} Los programas desarrollados son muy difíciles de usar y poco efectivos. \\ \hline
				\textbf{4.} Poco tiempo para asimilar los pasos a seguir. \\ \hline
				\textbf{5.} Falta de conocimiento por parte de los componentes del equipo.\\ \hline
			\end{tabular}
			
				\ \\
				\ \\
			%\textbf{Alta}\\
			
			\begin{tabular}{|p{12cm}|}
				\hline
				\textbf{Riesgos con prioridad Alta}\\ \hline \hline
				\textbf{6.} Retraso en las entregas por mala planificación.\\ \hline 
				\textbf{7.} El producto no se ajusta a lo que el cliente necesita por falta de comunicación. \\ \hline
				\textbf{8.}  No entregar todo lo acordado en la planificación por falta de tiempo.\\ \hline
				\textbf{9.} El cliente no sabe qué funciones debe desarrollar el producto. \\ \hline
				\textbf{10.} Los programas proporcionados son muy difíciles de usar y poco efectivos. \\ \hline
				\textbf{11.} La interfaz de usuario es demasiado difícil de construir. \\ \hline
				\textbf{12.} El cliente considera que la interfaz es difícil de usar.		\\ \hline
				\textbf{13.} Al cliente no le resulta atractiva la interfaz de usuario.			\\ \hline
				\textbf{14.} El cliente decide cambiar por completo la interfaz de usuario.			\\ \hline
				\textbf{15.} El cliente no sabe qué espera que haga el producto.			\\ \hline
				\textbf{16. }Los recursos son proporcionados demasiado tarde. 				\\ \hline
				\textbf{17. }Poco tiempo para realizar correciones. \\ \hline
				\textbf{18. }Nuestro producto no cumple con los requisitos de rendimiento. \\ \hline
				\textbf{19. }Nuestro producto no garantiza la calidad de uso. \\ \hline
				\textbf{20. }Poco tiempo para realizar el proyecto. \\ \hline

			\end{tabular}
				\ \\
				\ \\
			%\textbf{Media}\\
			
			\begin{tabular}{|p{12cm}|}
				\hline
				\textbf{Riesgos con prioridad Media}\\ \hline \hline
				\textbf{21.} Baja definitiva de algún miembro del equipo por enfermedad.\\ \hline 
				\textbf{22.} Abandono de la asignatura por parte de algún miembro del equipo.\\ \hline
				\textbf{23.} Abandono del proyecto por algún miembro del equipo. \\ \hline
				\textbf{24.} Abandono de la carrera por algún miembro del equipo. \\ \hline
				\textbf{25.} Las funciones son ineficientes.\\ \hline
				\textbf{26.} Poca calidad de las funciones y propiedades realizadas. \\ \hline
				\textbf{27.} Las distintas partes del proyecto no cumplen con su cometido \\ \hline
				\textbf{28.} Dificultad para hacer que las distintas funciones del proyecto se coordinen entre ellas. \\ \hline
				\textbf{29.} El producto no funciona debidamente en la plataforma en que se quiere usar. \\ \hline
				\textbf{30.} Falta de recursos para el desarrollo de la interfaz. \\ \hline
				\textbf{31.} El cliente cambia de opinión acerca de lo que debe hacer el proyecto. \\ \hline
				\textbf{32.} Falta de recursos para realizar el proyecto \\ \hline
			\end{tabular}
			
				\ \\
				\ \\
			%\textbf{Baja}\\
			
			\begin{tabular}{|p{12cm}|}
				\hline
				\textbf{Riesgos con prioridad Baja}\\ \hline \hline
				\textbf{33.} Baja del supervisor del proyecto.\\ \hline 
				\textbf{34.} El lenguaje no permite realizar todas las funciones.\\ \hline
				\textbf{35.} Las librerías de java no son eficientes para nuestro proyecto. \\ \hline
				%\textbf{34.} Poco tiempo para asimilar los pasos a seguir. \\ \hline
				%\textbf{35.} Falta de conocimiento por parte de los componentes del equipo.\\ \hline
			\end{tabular}
			\ \\
			\ \\
			\ \\
			\ \\
			Por lo tanto, los riesgos que serán tratados son los siguientes:
			\begin{itemize}
			  \item Baja temporal de algún miembro del equipo por enfermedad.
			  \item El cliente rechaza las funciones que hemos desarrollado.
			  \item Los programas desarrollados son muy difíciles de usar y poco efectivos.
			  \item Poco tiempo para asimilar los pasos a seguir. 
			  \item Falta de conocimiento por parte de los componentes del equipo. 
			  \item Retraso en las entregas por mala planificación. 
			  \item El producto no se ajusta a lo que el cliente necesita por falta de comunicación. 
			\end{itemize}
			

%El producto no funciona debidamente en la plataforma en que se quiere usar
 
\newpage
\mbox{}
\thispagestyle{empty}						% Hoja en blanco, sin numeros ni nada
\newpage
 
\part{Gestión del riesgo}
%\setcounter{section}{0}

	En esta parte se tratarán los riesgos que han sido elegidos por su prioridad en la sección anterior.\\
	Para ello, se analizarán diversos aspectos, que se muestran a continuación.
	%\section*{Baja temporal de algún miembro del equipo por enfermedad}
	%\begin{itemize}
	%\item \textbf{Nombre del riesgo}			% Hay que rellenar los siguientes campos de cada riesgo
		\begin{itemize}
			\item \textbf {Indicios de que se produzca: }Hechos que nos pueden llevar a pensar que es posible que ocurra
							el riesgo.
			\item \textbf {Mitigación: }En este momento el riesgo ya se ha producido. Aquí se describen las acciones
							que se deben llevar a cabo para minimizar las consecuencias, desde el momento en el que
							se produce el riesgo hasta el momento en el que se pone en marcha la acción de contingencia.
			\item \textbf {Acción de contingencia: }Acciones que se deben llevar a cabo una vez que el riesgo haya
							ocurrido, para que pueda ser resuelto.
			\item \textbf {Prevención: }Acciones que se deben realizar para evitar que el riesgo se produzca.
			
		\end{itemize}
	%\end{itemize}
	A continuación se exponen todos los riesgos:\\ \ \\ \ \\
	\setcounter{section}{0}
	\section{Baja temporal de algún miembro del equipo por enfermedad}
		\begin{itemize}
			\item \textbf {Indicios de que se produzca: }
				\begin{itemize}
				  \item  Periodos de gripe.
				  \item  Los miembros del equipo se encuentran cansados por sobrecarga de trabajo.
				  \item  Desmotivación en el trabajo, cansancio frecuente.
				\end{itemize} 
			\item \textbf {Mitigación: }
				\begin{itemize}
					\item Repartir el trabajo de forma equitativa entre los restantes miembros del grupo.
					\item En caso de que esté de baja por una enfermedad contagiosa, comprobar que el resto de el equipo no ha sido contagiado.
					%\item 
				\end{itemize} 
			\item \textbf {Acción de contingencia: }
				\begin{itemize}
					\item Suplir la baja contratando a otro trabajador.
				\end{itemize}
			\item \textbf {Prevención: }
				\begin{itemize}
				  \item La enfermedad puede ser imprevisible. 
				  \item No sobrecargar a los trabajadores. Tener cuidado en temporadas de gripe. 
				\end{itemize} 
			\end{itemize}
		
	\section{El cliente rechaza las funciones que hemos desarrollado}
		\begin{itemize}
			\item \textbf {Indicios de que se produzca: }
				\begin{itemize}
				  \item El cliente discute todas las
						propuestas que le ofrecemos y ninguna de nuestras posibles soluciones le parece
						adecuada.
				\end{itemize}
			\item \textbf {Acción de contingencia: }
				\begin{itemize}
				  \item Reunirnos de nuevo con el cliente,
						analizar lo que nos pide y llegar a un acuerdo para hacer una síntesis entre las
						funciones que ya hemos desarrollado y lo que el cliente pide.
				\end{itemize}
			\item \textbf {Prevencion: }
				\begin{itemize}
				  \item Hacer más reuniones con el cliente para conocer
						mejor su idea, es decir, qué quiere y cómo lo quiere, para conocer todos los
						detalles que debe cumplir el producto. 
				  \item Una vez conocidos los detalles a fondo
						comenzar algunas funciones y a medida que se van desarrollando mostrárselas al
						cliente para que de su opinión y así realizar un trabajo que sea de su agrado.
				\end{itemize}
			\item \textbf {Mitigacion: }
				\begin{itemize}
			  			\item Modificar las funciones hasta que sean del gusto del
							cliente.
				\end{itemize}
		\end{itemize}
	
	%
	\section{Los programas desarrollados son muy difíciles de usar y poco efectivos}
		\begin{itemize}
			\item \textbf {Indicios de que se produzca: }
				\begin{itemize}
				  \item Hay poca visibilidad del producto.
				  \item Los requisitos son poco claros. 
				  \item El equipo no tiene claro lo que hay que hacer.
				  \item El equipo no sabe qué es exactamente lo que el cliente desea. 
				  \item No se sabe qué pasos hay que seguir debido a una mala planificación y un mal diseño.
				\end{itemize} 
			\item \textbf {Mitigación: }
				\begin{itemize}
				  \item Hablar con el cliente para saber dónde está el problema.
				  \item Probar, monitorizar y depurar el programa, para ver qué partes son problemáticas y qué partes son 
				  		poco efectivas.
				\end{itemize}
			\item \textbf {Acción de contingencia: }
				\begin{itemize}
				  \item Rehacer las partes problemáticas.
				  \item Comprobar que los problemas se han resuelto.
				  \item Entregar al cliente la nueva versión del programa. 
				\end{itemize} 
			\item \textbf {Prevención: }
				\begin{itemize}
				  \item Elaborar prototipos para que el programa pueda ser probado y corregido.
				  \item Mejorar la planificación, aumentar la claridad de los requisitos.
				  \item Hablar más con el cliente, para tener claro qué es lo que él quiere.
				\end{itemize}
			\end{itemize}
		
	\section{Poco tiempo para asimilar los pasos a seguir}
		\begin{itemize}
			\item \textbf {Indicios de que se produzca: }Vamos retrasados en las clases y el
			profesor tiene que explicar más rápido, o se aproxima una época de mucho trabajo
			y exámenes por lo que no podemos dedicar tanto tiempo a la asignatura y al
			proyecto.
			\item \textbf {Acción de contingencia: }
				\begin{itemize}
				  \item Repartir bien el trabajo.
				  \item Marcarnos fechas para realizar cada parte muy cortas y ser muy estrictos. 
				  \item Para ello debemos dedicar más tiempo al proyecto. De este modo llevaremos una información
						muy actualizada de que debemos llevar hecho, cómo hacerlo y todo el grupo
						conocerá el estado del proyecto. 
				  \item También podríamos dedicar un tiempo a ayudarnos
						a comprender mejor lo que debemos saber de la asignatura.
				\end{itemize}
			\item \textbf {Prevención: }
				\begin{itemize}
				  \item Llevar las clases de Ingeniería del Software al día, incluso estudiar por adelantado. 
				  \item Comprender todo antes de comenzar con el
						proyecto y dedicarnos a éste desde el principio, cuidando no dejarlo todo para
						cuando se aproxime la fecha de entrega.  % No me lo creo ni yo
				\end{itemize}
			\item \textbf {Mitigación: }Dedicar nuestro tiempo únicamente a estudiar la
			asignatura y a realizar el proyecto.
		\end{itemize}

	
	\section{Falta de conocimiento por parte de los componentes del equipo}
		\begin{itemize}
			\item \textbf {Indicios de que se produzca: }%los miembros del equipo se encuentran desorientados, no saben qué hacer, avanzes torpes,
				\begin{itemize}
					\item Los miembros del equipo no saben qué hacer.
					\item Se hacen avanzes torpes e inseguros.
					\item Poca organización. 
					\item En determinados momentos parece que el proyecto se va a detener.
					%\item 
				\end{itemize}
			\item \textbf {Mitigación: }Recordemos que si el riesgo ocurre, esto nos llevaría a un empeoramiento de la calidad
										del producto, aumento de la dificultad del desarrollo y retrasos en las entregas.
				\begin{itemize}
				  \item Contratar o conseguir la colaboración de alguien con experiencia.
				  \item Analizar qué aspectos son los desconocidos. 
				  \item Corrección de errores por fallos de concepto.
				  \item Reorganizar las partes incorrectas del proyecto, esta vez teniendo conocimiento de cómo se deben
				  		de hacer las cosas.
				  \item Hablar con el cliente para acordar nuevas fechas. 
				\end{itemize} 
			\item \textbf {Acción de contingencia: }
				\begin{itemize}
				  \item Rehacer las partes incorrectas del proyecto.
				  \item Mejorar la calidad del producto.
				\end{itemize}
			\item \textbf {Prevención: }%
				\begin{itemize}
					\item Tener claros los conceptos de la IS.
					\item En caso de duda, consultar con gente
						con experiencia, para saber cómo actuar.
					\item En caso de que el equipo se encuentre desorientado o algún miembro del equipo no sepa qué hacer,
							pedir ayuda y consultar a alguien con experiencia en la IS. 
				\end{itemize} 
			\end{itemize}
	
	\section{Retraso en las entregas por mala planificación}
		\begin{itemize}
			\item \textbf {Indicios de que se produzca: }
				\begin{itemize}
				  \item Conforme avanza el proyecto la
						relacion entre esfuerzo real y planificación no se cumple, por tanto se necesita
						mas tiempo para realizar la misma tarea. Esto es lo que lleva a los retrasos.
				\end{itemize}
			\item \textbf {Mitigacion: }
				\begin{itemize}
				  \item Dedicar mas tiempo a la entrega hasta que se acabe,
					aunque eso implique menor calidad.
				\end{itemize}
			\item \textbf {Acción de contingencia: }
				\begin{itemize}
				  \item Volver a realizar una nueva
						planificación basándonos en la experiencia de los últimos hechos y con más
						cuidado que la última vez. 
				  \item Para cumplir con las entregas también será necesario
						aumentar la cantidad de horas dedicadas.
				\end{itemize}
			\item \textbf {Prevencion: }Realizar una planificación más extensa que indique
				en qué momento debemos tener cada parte, sin necesidad de que exista una
				entrega.
		
		\end{itemize}
	
	\section{El producto no se ajusta a lo que el cliente necesita por falta de comunicación}
		\begin{itemize}
			\item \textbf {Indicios de que se produzca: }
				\begin{itemize}
					\item En determinadas ocasiones, el equipo se encuentra desorientado acerca de lo que tiene que hacer.
					\item Existen diferentes posturas dentro del equipo acerca de lo que se tiene que hacer. 
					\item Hay poca comunicación con el cliente.
					\item En las reuniones con el cliente no se saca nada en claro. Hay ambigüedades.
					%\item 
				\end{itemize}
			\item \textbf {Mitigación: }
				\begin{itemize}
				  \item Detectar qué partes del programa son las que fallan.  
				  \item Hablar con el cliente para saber por qué el programa no se ajusta a sus necesidades. 
				  \item Estudiar a fondo aquellas características que no son necesarias en el programa, y aquellas
				  		que lo son y que no están presentes.
				\end{itemize} 
			\item \textbf {Acción de contingencia: }
				\begin{itemize}
				  \item  Mostrar al cliente qué características se cambiarán, para obtener el visto bueno.
				  \item Rehacer el programa, teniendo en cuenta la opinión del cliente, y haciendo los cambios necesarios.
				\end{itemize} 
			\item \textbf {Prevención: }
				\begin{itemize}
				  \item Evitar ambigüedades en la especificación.
				  \item Al hablar con el cliente, dejar claras todas las características que ha de tener el programa.  
				\end{itemize} 
			
		\end{itemize}
	
	
		
	%=======
%>>>>>>> .r75
	% En la diapositiva 29 hay un modelo para rellenar los campos de cada riesgo


\newpage
\mbox{}
\thispagestyle{empty}						% Hoja en blanco, sin numeros ni nada, al final del documento
\newpage

\end{document}

